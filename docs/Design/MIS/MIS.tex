\documentclass[12pt, titlepage]{article}

\usepackage{amsmath, mathtools}

\usepackage[round]{natbib}
\usepackage{amsfonts}
\usepackage{amssymb}
\usepackage{graphicx}
\usepackage{colortbl}
\usepackage{xr}
\usepackage{hyperref}
\usepackage{longtable}
\usepackage{xfrac}
\usepackage{tabularx}
\usepackage{float}
\usepackage{siunitx}
\usepackage{booktabs}
\usepackage{multirow}
\usepackage[section]{placeins}
\usepackage{caption}
\usepackage{fullpage}

\hypersetup{
bookmarks=true,     % show bookmarks bar?
colorlinks=true,       % false: boxed links; true: colored links
linkcolor=red,          % color of internal links (change box color with linkbordercolor)
citecolor=blue,      % color of links to bibliography
filecolor=magenta,  % color of file links
urlcolor=cyan          % color of external links
}

\usepackage{array}

\externaldocument{../../SRS/SRS}

%% Comments

\usepackage{color}

\newif\ifcomments\commentstrue

\ifcomments
\newcommand{\authornote}[3]{\textcolor{#1}{[#3 ---#2]}}
\newcommand{\todo}[1]{\textcolor{red}{[TODO: #1]}}
\else
\newcommand{\authornote}[3]{}
\newcommand{\todo}[1]{}
\fi

\newcommand{\wss}[1]{\authornote{blue}{SS}{#1}} 
\newcommand{\plt}[1]{\authornote{magenta}{TPLT}{#1}} %For explanation of the template
\newcommand{\an}[1]{\authornote{cyan}{Author}{#1}}


\newcommand{\famname}{Lattice Boltzmann Solvers}

\newcounter{mnum}
\newcommand{\mthemnum}{M\themnum}

\begin{document}

\title{Module Interface Specification for \famname}

\author{Peter Michalski}

\date{\today}

\maketitle

\pagenumbering{roman}

\section{Revision History} \label{MISREVHISTORY}

\begin{tabularx}{\textwidth}{p{3cm}p{2cm}X}
\toprule {\bf Date} & {\bf Version} & {\bf Notes}\\
\midrule
Nov. 25, 2019 & 1.0 & Initial Document\\
\bottomrule
\end{tabularx}

~\newpage

\section{Symbols, Abbreviations and Acronyms}

See CA Documentation for \famname \ \citep{LBM_CA_PM}.

\newpage

\tableofcontents

\newpage

\pagenumbering{arabic}

\section{Introduction}

The following document details the Module Interface Specifications for \famname
, which provides a library of services based on Lattice Boltzmann Methods (LBM).
LBM are a family of fluid dynamics algorithms for simulating single-phase and
multiphase fluid flows, often incorporating additional physical complexities
\citep{chen1998lattice}.

Complementary documents include the System Requirement Specifications
and Module Guide.  The full documentation and implementation can be
found \href{https://github.com/peter-michalski/LatticeBoltzmannSolvers}{here} \citep{LBM_PM}.

\section{Notation}

The structure of the MIS for modules comes from \citet{HoffmanAndStrooper1995},
with the addition that template modules have been adapted from
\cite{GhezziEtAl2003}.  The mathematical notation comes from Chapter 3 of
\citet{HoffmanAndStrooper1995}.  For instance, the symbol := is used for a
multiple assignment statement and conditional rules follow the form $(c_1
\Rightarrow r_1 | c_2 \Rightarrow r_2 | ... | c_n \Rightarrow r_n )$.

The following table summarizes the primitive data types used by \famname. 

\begin{center}
\renewcommand{\arraystretch}{1.2}
\noindent 
\begin{tabular}{l l p{7.5cm}} 
\toprule 
\textbf{Data Type} & \textbf{Notation} & \textbf{Description}\\ 
\midrule
Boolean & Boolean & true or false\\
natural number & $\mathbb{N}$ & a number without a fractional component in [1, $\infty$) \\
real & $\mathbb{R}$ & any number in (-$\infty$, $\infty$)\\
string & string & single or multiple symbols or digits\\
\bottomrule
\end{tabular} 
\end{center}

\noindent
The specification of \famname \ uses some derived data types: sequences, strings, and
tuples. Sequences are lists filled with elements of the same data type. Strings
are sequences of characters. Tuples contain a list of values, potentially of
different types. In addition, \famname \ uses functions, which
are defined by the data types of their inputs and outputs. Local functions are
described by giving their type signature followed by their specification.

~\newpage

\section{Module Decomposition}

The following table is taken directly from the Module Guide document for this project.

\begin{table}[h!]
	\centering
	\begin{tabular}{p{0.3\textwidth} p{0.6\textwidth}}
		\toprule
		\textbf{Level 1} & \textbf{Level 2}\\
		\midrule
		
		{Hardware-Hiding Module}
		& \\
		\midrule
		
		\multirow{7}{0.3\textwidth}{Behaviour-Hiding Module}
		& M2: System Control Module\\
		& M3: Input Reading Module\\
		& M4: Input Checking Module\\
		& M5: LBM Control Module\\
		& M6: Streaming Module\\
		& M7: Collision Module\\ 
		& M8: Problem Module\\
		& M9: Lattice Module\\
		& M10: Boundary Module\\
		\midrule
		
		\multirow{1}{0.3\textwidth}{Software Decision Module}
		& M11: Image Rendering Module\\
		& M12: Data Structure Module\\
		& M13: Input Types Module\\
		\bottomrule
		
	\end{tabular}
	\caption{Module Hierarchy}
	\label{TblMH}
\end{table}


~\newpage

\section{MIS of M2: System Control Module} \label{SCModule}

The secret of this module is the algorithm to control \famname.

\subsection{Module}

SystemControl

\subsection{Uses}

\begin{itemize}
	\item Hardware Hiding
	\item Input Reading (Section \ref{IRModule})
	\item LBM Control (Section \ref{LCModule})
	\item Problem Parameter (Section \ref{PRModule})
	\item Image Rendering (Section \ref{OUModule})
	\item Data Structure (Section \ref{DSModule})
\end{itemize}

\subsection{Syntax}

\subsubsection{Exported Constants}
N/A

\subsubsection{Exported Access Programs}
N/A

\subsection{Semantics}

\subsubsection{State Variables}
inputData: DataStructure\\
problemData: $\mathbb{R}$ \& $\mathbb{N}$ \wss{What does this notation mean?
  This does not look like proper mathematics.}\\ 
imageData: array[$\mathbb{R}$]\\
imageOut: PNG\\


\subsubsection{Environment Variables}

N/A

\subsubsection{Assumptions}

The user has run the {\famname} program.  \wss{I don't understand this assumption.}

\subsubsection{Access Routine Semantics}

N/A

\subsubsection{Local Functions}

\begin{center}
	\begin{tabular}{p{4cm} p{4cm} p{2cm} p{2cm}}
		\hline
		\textbf{Name} & \textbf{In} & \textbf{Out} & \textbf{Exceptions} \\
		\hline
		mainFunction & - & - &- \\
		\hline
	\end{tabular}
\end{center}

\subsubsection{Local Function Semantics}

\noindent mainFunction():
\begin{itemize}
	\item transition: $out$ := N/A
	\item exception: N/A
\end{itemize}

The function calls other modules of {\famname} to solve the given problem. The
function calls InputReading.inputArray(), followed by
ProblemParameter.formatInput() and LBMControl.performLBM(). Finally the module
calls ImageRendering.imageFunc() and sends the output of that function to the
hardware. \wss{This is vague.  Can you write this as pseudo code?}


~\newpage

\section{MIS of M3: Input Reading Module} \label{IRModule} 

The secret of this module is the algorithm that gathers the input data.

\subsection{Module}

InputReading

\subsection{Uses}

\begin{itemize}
	\item Hardware Hiding
	\item Input Checking (Section \ref{ICModule})
	\item Data Structure (Section \ref{DSModule})
\end{itemize}

\subsection{Syntax}

\subsubsection{Exported Constants}
N/A

\subsubsection{Exported Access Programs}

\begin{center}
	\begin{tabular}{p{2cm} p{4cm} p{3cm} p{2cm}}
		\hline
		\textbf{Name} & \textbf{In} & \textbf{Out} & \textbf{Exceptions} \\
		\hline
		inputArray & - & inputData: DataStructure & NotFound, ErrorRead \\
		\hline
	\end{tabular}
\end{center}

\subsection{Semantics}

\subsubsection{State Variables}
N/A

\subsubsection{Environment Variables}

InputLocation (string): ``./Input/input.txt'' \wss{The environment variable is
  the file, not the files name.  Have a look at the example at
  \url{https://gitlab.cas.mcmaster.ca/smiths/se2aa4_cs2me3/blob/master/Assignments/PreviousYears/2018/A2/A2.pdf}}

\subsubsection{Assumptions}

The System Control Module M2 (Section \ref{SCModule}) has called InputReading.inputArray().

\subsubsection{Access Routine Semantics}

\noindent inputArray():
\begin{itemize}
	\item transition: $out$ := +(inputData[key][value] $\Leftarrow$
          ./Input/input.txt) \wss{I have no idea what this notation means.  A
            transition field does not have an output.  The left pointing arrow
            is not a notation that Hoffman and Strooper use.}
	\item exception: input.txt $\notin$ ./Input/ $\Rightarrow$ NotFound
	\item exception: (inputData $\Leftarrow$ ?) $\Rightarrow$ ErrorRead
          \wss{You should use a conditional rule to combine all of your
            exceptions in one rule.  Two exception fields is confusing, and not
            part of the template.}
\end{itemize}

The function will read all lines from the input file and place each value into inputData.

\subsubsection{Local Functions}
N/A

\newpage

\section{MIS of M4: Input Checking Module} \label{ICModule} 

This secret of this module is the algorithm that checks if input values fall within allowable parameters.

\subsection{Module}

InputChecking

\subsection{Uses}

\begin{itemize}
	\item Data Structure (Section \ref{DSModule})
\end{itemize}

\subsection{Syntax}

\subsubsection{Exported Constants}
N/A

\subsubsection{Exported Access Programs}

\begin{center}
	\begin{tabular}{p{2cm} p{3cm} p{3cm} p{2.5cm}}
		\hline
		\textbf{Name} & \textbf{In} & \textbf{Out} & \textbf{Exceptions} \\
		\hline
		verifyInputs & inputData: DataStructure & Boolean & OutBounds, UnknwnParm \\
		\hline
	\end{tabular}
\end{center}

\wss{The DataStructure module documents an abstract object, but here you are
  using DataStructure as a type.  If it is an ADT, it should be documented
  differently.  If it is an abstract object, you cannot pass it as a parameter;
  you simply use it in that case.}
\subsection{Semantics}

\subsubsection{State Variables}
The set of acceptableRanges is:\\
\textit{LIBRARY}: set of $\mathbb{N}$: \{1\} - Libraries are associated by
numbers in the program. The numbers are available for reference in the User
Guide. \wss{What does this notation mean?  Is the entry after the second full
  colon the value of the state variable?  Is it the initial value of the state
  variable, or is this a constant?}\\
\textit{PROBLEM}: set of $\mathbb{N}$: \{1\} - Problems are associated by
numbers in the program. The numbers are available for reference in the User
Guide.\\
\textit{DIMENSIONS}: set of $\mathbb{N}$: \{2\}\\
\textit{VEL\_DIRS}: set of $\mathbb{N}$: \{9\}\\
\textit{REYNOLDS\_MIN}: $\mathbb{R}$: \{0.001\}\\
\textit{REYNOLDS\_MAX}: $\mathbb{R}$: \{5000\}\\
\textit{DENSITY\_MIN}: $\mathbb{R}$: \{0.0708\}\\
\textit{DENSITY\_MIN}: $\mathbb{R}$: \{13.6\}\\
\textit{BULK\_VIS\_MIN}: $\mathbb{R}$: \{0.0001\}\\
\textit{BULK\_VIS\_MIN}: $\mathbb{R}$: \{20000\}\\
\textit{SHEAR\_VIS\_MIN}: $\mathbb{R}$: \{0.001\}\\
\textit{SHEAR\_VIS\_MIN}: $\mathbb{R}$: \{20000\}\\
\textit{TIME\_MIN}: $\mathbb{N}$: \{1\}\\

\subsubsection{Environment Variables}
N/A

\subsubsection{Assumptions}

The Input Reading Module M3 (Section \ref{IRModule}) has called
InputChecking.verifyInputs().  \wss{This is confusing.  This module assumes that
  this module has already been called?  This sounds like unwanted recursion in
  the documentation.}

\subsubsection{Access Routine Semantics}

\noindent verifyInputs(inputData):
\begin{itemize}
	\item output: (inputData[value] $!>$ acceptableRanges) $\land$ (inputData[value] $!<$ acceptableRanges) $\Rightarrow$ return True
	\item exception: (inputData[value] $>$ acceptableRanges) $\lor$ (inputData[value] $<$ acceptableRanges) $\Rightarrow$ OutBounds
	\item exception: inputData[key] $\notin$ inputTypes $\Rightarrow$ UnknwnParm
\end{itemize} \wss{Where does value come from?  This variable is not bound to
  anything in this expression.  Do you want a quantification over all values?}

The function will iterate through each inputValue InputTypes key and check if
the key is known to the program, as well as check if associated values of known
keys fall within an acceptable range of the state variables.

\subsubsection{Local Functions}

N/A

\newpage

\section{MIS of M5: LBM Control Module} \label{LCModule}

The secret of this module is the algorithm which controls the LBM library.

\subsection{Module}

LBMControl

\subsection{Uses}

\begin{itemize}
	\item Streaming (Section \ref{STModule})
	\item Collision (Section \ref{COModule})
\end{itemize}

\subsection{Syntax}

\subsubsection{Exported Constants}

\subsubsection{Exported Access Programs}

\begin{center}
	\begin{tabular}{p{2cm} p{3cm} p{4cm} p{2cm}}
		\hline
		\textbf{Name} & \textbf{In} & \textbf{Out} & \textbf{Exceptions} \\
		\hline
		performLBM & inputData: DataStructure, problemData: $\mathbb{R}$ \& $\mathbb{N}$ & imageData: array[$\mathbb{R}$] & - \\
		\hline
	\end{tabular}
\end{center}

\wss{You need to decide whether DataStructure is an abstract object, or an
  abstract data type.}
\wss{I think (?)  you are using problemData to hold real and natural number
  data.  You need to be more specific about the type of problemData.  Is it a
  tuple with named fields?  I know you want to leave things open for different
  uses.  In that case you could define specific options for the problemData type
  and instantiate the one in this spec that makes sense for the program family
  member in use.  To keep things simple for the final implementation, maybe you
  can just pick a type of problem and document the fields of the problemData
  type specifically for that type of problem?}

\subsection{Semantics}

\subsubsection{State Variables}
imageData: array[$\mathbb{R}$] %\wss{the image data is a single real number?} %\pm{No, it is an array. I have updated the variable to reflect this.} \wss{The LBM calculations output an image?  I would much rather the output is the velocity (or whatever CFD parameter is of interest).  This data can then be turned into an image by another routine, or simply returned to the user, and they can make their own image.  An image as output is not very reusable in another context.} %\pm{The library and problem that will first be implemented will show the image, future additions to the library will include additional numerical output}

\subsubsection{Environment Variables}
N/A

\subsubsection{Assumptions}

The System Control Module M2 (Section \ref{SCModule}) has called
LBMControl.performLBM().  \wss{This recursive assumption is showing up again.
  :-)  I think this is an assumption you want in the System Control Module, not here?}

\subsubsection{Access Routine Semantics}

\noindent performLBM(inputData, problemData):
\begin{itemize}
	\item transition: $out$ := imageData = $\mathbb{R}$ $\Leftarrow$
          Streaming.streamingFunc(), $\mathbb{R}$ $\Leftarrow$
          Collision.collisionFunc() \wss{transitions change the state, outputs
            output values. I don't know which you are trying to do here.  The
            left arrows are also confusing.  If you don't know a notation for
            what you are trying to say, you are better off saying it in words,
            than inventing a notation.}
\item exception: N/A
\end{itemize}

The function will calculate the vorticity vector values, iterating through each velocity direction, calling the streaming and collision module functions.

\subsubsection{Local Functions}

N/A

\newpage

\section{MIS of M6: Streaming Module} \label{STModule}

The secret of this module is the algorithm to calculate the streaming pf particles.

\subsection{Module}

Streaming

\subsection{Uses}
N/A
\subsection{Syntax}

\subsubsection{Exported Constants}
N/A

\subsubsection{Exported Access Programs}

\begin{center}
	\begin{tabular}{p{3cm} p{3cm} p{3cm} p{2cm}}
		\hline
		\textbf{Name} & \textbf{In} & \textbf{Out} & \textbf{Exceptions} \\
		\hline
		streamingFunc & Velocityi: $\mathbb{R}$, Time: $\mathbb{N}$, PositionVector: $\mathbb{R}$, Force: $\mathbb{R}$, Mass: $\mathbb{R}$ & $\mathbb{R}$ & NAN \\
		\hline
	\end{tabular}
\end{center}

\subsection{Semantics}

\subsubsection{State Variables}

N/A

\subsubsection{Environment Variables}
maxVariableSize = The maximum allowable value held in the variable.\\
minVariableSize = The minimum allowable value held in the variable.
\wss{These are not environment variables.}

\subsubsection{Assumptions}

The LBM Control Module M5 (Section \ref{LCModule}) has called streaming.streamingFunc().

\subsubsection{Access Routine Semantics}

\noindent streamingFunc(Velocityi, Time, PositionVector, Force, Mass):
\begin{itemize}
	\item transition: $out$ := $f_i(\mathrm{\textbf{x}} +\mathrm{e}_i dt, t + dt) - f_i(\mathrm{\textbf{x}},t)$
	\item exception: ($f_i(\mathrm{\textbf{x}} +\mathrm{e}_i dt, t + dt) - f_i(\mathrm{\textbf{x}},t)$ $>$ maxVariableSize) $\lor$ ($f_i(\mathrm{\textbf{x}} +\mathrm{e}_i dt, t + dt) - f_i(\mathrm{\textbf{x}},t)$ $<$ minVariableSize) $\Rightarrow$ NAN
\end{itemize}

\wss{What is the connection between the parameters of the streaming function and
  the equation given in the transition?  What is the mapping between x, e,
  Velocityi, etc?  You should use the same symbols in the input as you use for
  the transition.}
\wss{The confusion between transition and output has to be addressed throughout
  the document.}
The function calculates the streaming step value for each velocity direction (i).

\subsubsection{Local Functions}

\begin{center}
	\begin{tabular}{p{2cm} p{3cm} p{4cm} p{2cm}}
		\hline
		\textbf{Name} & \textbf{In} & \textbf{Out} & \textbf{Exceptions} \\
		\hline
		pdfFunc ($f(\mathrm{\textbf{x}},\mathrm{e},t)$) & Velocityi: $\mathbb{R}$, Time: $\mathbb{N}$, PositionVector: $\mathbb{R}$, Force: $\mathbb{R}$, Mass: $\mathbb{R}$ & $\mathbb{R}$ & NAN \\
		\hline
	\end{tabular}
\end{center}

\subsubsection{Local Function Semantics}

\noindent pdfFunc(Velocityi, Time, PositionVector, Force, Mass):
\begin{itemize}
	\item transition: $out$ := $f(\mathrm{\textbf{x}},\mathrm{e},t) = f(\mathrm{\textbf{x}} +\mathrm{e}dt, \mathrm{e} + \frac{F}{kg}dt, t + dt)$
	\item exception: ($f(\mathrm{\textbf{x}} +\mathrm{e}dt, \mathrm{e} + \frac{F}{kg}dt, t + dt)$) $>$ maxVariableSize) $\lor$ ($f(\mathrm{\textbf{x}} +\mathrm{e}dt, \mathrm{e} + \frac{F}{kg}dt, t + dt)$) $<$ minVariableSize) $\Rightarrow$ NAN
\end{itemize}

The function finds the probability that a particle is at position $\textbf{x}$ and has velocity $\mathrm{e}$ at time t.\\

\wss{Where is the local function used in the specification of Streaming Module?
  Local functions are only used within the spec of one module.  They are not
  used elsewhere, or part of the required implementation.}
\newpage

\section{MIS of M7: Collision Module} \label{COModule} 

The secret of this module is the algorithm to calculate the collision of particles.

\subsection{Module}

Collision

\subsection{Uses}
N/A

\subsection{Syntax}

\subsubsection{Exported Constants}
N/A

\subsubsection{Exported Access Programs}

\begin{center}
	\begin{tabular}{p{3cm} p{3cm} p{3cm} p{2cm}}
		\hline
		\textbf{Name} & \textbf{In} & \textbf{Out} & \textbf{Exceptions} \\
		\hline
		collisionFunc & RelaxationRate : $\mathbb{R}$, Weighti: $\mathbb{R}$, Density: $\mathbb{R}$, UnitVector: $\mathbb{R}$, MacroscopicVelocity: $\mathbb{R}$, SpeedSound: $\mathbb{R}$, VelocityDirection: $\mathbb{N}$ & $\mathbb{R}$ & NAN \\
		\hline
	\end{tabular}
\end{center}

\subsection{Semantics}

\subsubsection{State Variables}
N/A

\subsubsection{Environment Variables}
maxVariableSize = The maximum allowable value held in the variable.\\
minVariableSize = The minimum allowable value held in the variable.

\subsubsection{Assumptions}

The LBM Control Module M5 (Section \ref{LCModule}) has called Collision.streamingFunc().

\subsubsection{Access Routine Semantics}

\noindent collisionFunc(RelaxationRate, Weighti, Density, UnitVector, MacroscopicVelocity, SpeedSound, VelocityDirection):
\begin{itemize}
	\item transition: $out$ := $\frac{1}{\tau}(f_{i}^{eq}-f_{i})$
	\item exception: ($\frac{1}{\tau}(f_{i}^{eq}-f_{i})$ $>$ maxVariableSize) $\lor$ ($\frac{1}{\tau}(f_{i}^{eq}-f_{i})$ $<$ minVariableSize) $\Rightarrow$ NAN
\end{itemize}

\wss{I do not see the mapping between the parameters and the transition
  equation.  I won't continue to write this comment, but it applies throughout
  this document.}

The function calculates the collision step value for each velocity direction (i).

\subsubsection{Local Functions}

\begin{center}
	\begin{tabular}{p{2cm} p{3cm} p{4cm} p{2cm}}
		\hline
		\textbf{Name} & \textbf{In} & \textbf{Out} & \textbf{Exceptions} \\
		\hline
		edfFunc ($f_{i}^{eq})$ & Weighti: $\mathbb{R}$, Density: $\mathbb{R}$, UnitVector: $\mathbb{R}$, MacroscopicVelocity: $\mathbb{R}$, SpeedSound: $\mathbb{R}$, VelocityDirection: $\mathbb{N}$ & $\mathbb{R}$ & NAN \\
		relFunc ($f_{i}$) & Weighti: $\mathbb{R}$, Density: $\mathbb{R}$, UnitVector: $\mathbb{R}$, MacroscopicVelocity: $\mathbb{R}$, SpeedSound: $\mathbb{R}$, VelocityDirection: $\mathbb{N}$ & $\mathbb{R}$  & NAN \\
		\hline
	\end{tabular}
\end{center}

\subsubsection{Local Function Semantics}

\noindent edfFunc(Weighti, Density, UnitVector, MacroscopicVelocity, SpeedSound, VelocityDirection):
\begin{itemize}
	\item transition: $out$ := $f_{i}^{eq} = w_{i}p(1 + 3 \cdot \frac{\textbf{e}_\textbf{i}\cdot\textbf{u}}{c_s^2}+\frac{9}{2} \cdot \frac{(\textbf{e}_\textbf{i}\cdot\textbf{u})^2}{c_s^4}-\frac{3}{2} \cdot \frac{\textbf{u}\cdot\textbf{u}}{c_s^4})$
	\item exception: ($w_{i}p(1 + 3 \cdot \frac{\textbf{e}_\textbf{i}\cdot\textbf{u}}{c_s^2}+\frac{9}{2} \cdot \frac{(\textbf{e}_\textbf{i}\cdot\textbf{u})^2}{c_s^4}-\frac{3}{2} \cdot \frac{\textbf{u}\cdot\textbf{u}}{c_s^4})$ $>$ maxVariableSize) $\lor$ ($w_{i}p(1 + 3 \cdot \frac{\textbf{e}_\textbf{i}\cdot\textbf{u}}{c_s^2}+\frac{9}{2} \cdot \frac{(\textbf{e}_\textbf{i}\cdot\textbf{u})^2}{c_s^4}-\frac{3}{2} \cdot \frac{\textbf{u}\cdot\textbf{u}}{c_s^4})$ $<$ minVariableSize) $\Rightarrow$ NAN
\end{itemize}

The function calculates the equilibrium distribution for each velocity direction (i).\\

\noindent relFunc(Weighti, Density, UnitVector, MacroscopicVelocity, SpeedSound, VelocityDirection):
\begin{itemize}
	\item transition: $out$ := $f_{i} = f_{i} - \frac{1}{\tau}(f_{i}^{eq}-f_{i})$
	\item exception: ($f_{i} - \frac{1}{\tau}(f_{i}^{eq}-f_{i})$ $>$ maxVariableSize) $\lor$ ($f_{i} - \frac{1}{\tau}(f_{i}^{eq}-f_{i})$ $<$ minVariableSize) $\Rightarrow$ NAN
\end{itemize}

The function calculates the relaxation update for each velocity direction (i).\\

\newpage

\section{MIS of M8: Problem Module} \label{PRModule}

The secret of this module is the structure of the LBM input parameters.

\subsection{Module}

ProblemParameter

\subsection{Uses}

\begin{itemize}
	\item Lattice (Section \ref{LAModule})
	\item Boundary (Section \ref{BOModule})
	\item Data Structure (Section \ref{DSModule})
\end{itemize}

\subsection{Syntax}

\subsubsection{Exported Constants}
N/A

\subsubsection{Exported Access Programs}

\begin{center}
	\begin{tabular}{p{2cm} p{3cm} p{4cm} p{2cm}}
		\hline
		\textbf{Name} & \textbf{In} & \textbf{Out} & \textbf{Exceptions} \\
		\hline
		formatInput & inputData: DataStructure & problemData: $\mathbb{R}$ \& $\mathbb{N}$ & - \\
		\hline
	\end{tabular}
\end{center}

\subsection{Semantics}

\subsubsection{State Variables}
N/A

\subsubsection{Environment Variables}
N/A

\subsubsection{Assumptions}

The System Control Module M2 (Section \ref{SCModule}) has called ProblemParameter.formatInput().

\subsubsection{Access Routine Semantics}

\noindent formatInput(inputData):
\begin{itemize}
	\item transition: $out$ := problemData = $\mathbb{N}$ $\in$ inputData, $\mathbb{R}$ $\Leftarrow$ Boundary.getBoundary(), $\mathbb{R}$ $\Leftarrow$ Lattice.getWeights()
	\item exception: N/A
\end{itemize}

The function will set up the structure for the LBM input parameters based on the library that the user has requested. The function will use the Lattice and Boundary module functions.

\subsubsection{Local Functions}
N/A

\newpage

\section{MIS of M9: Lattice Module} \label{LAModule} 

The secret of this module is the structure of the lattice model.

\subsection{Module}

Lattice

\subsection{Uses}
\begin{itemize}
	\item Data Structure (Section \ref{DSModule})
\end{itemize}

\subsection{Syntax}

\subsubsection{Exported Constants}
parameterValues := (9, 4/9, 1/9, 1/9, 1/9, 1/9, 1/36, 1/36, 1/36, 1/36) 

The above holds the coefficient weights for velocity directions of a Q9 model.

\subsubsection{Exported Access Programs}

\begin{center}
	\begin{tabular}{p{2cm} p{3cm} p{4cm} p{2cm}}
		\hline
		\textbf{Name} & \textbf{In} & \textbf{Out} & \textbf{Exceptions} \\
		\hline
		getWeights & inputData: DataStructure & $\mathbb{R}$ & NoLattice \\
		\hline
	\end{tabular}
\end{center}

\subsection{Semantics}

\subsubsection{State Variables}
N/A

\subsubsection{Environment Variables}
N/A

\subsubsection{Assumptions}

The Problem Module M8 (Section \ref{PRModule}) has called Lattice.getWeights().

\subsubsection{Access Routine Semantics}

\noindent getWeights(inputData):
\begin{itemize}
	\item transition: $out$ := Weighti ($\mathbb{R}$) $\in$ parameterValues
	\item exception: inputData[VelocityDirections] $\notin$ parameterValues $\Rightarrow$ NoLattice
\end{itemize}

The function sets coefficient weight data for the input lattice model, and returns an error if the weight data is not available for the input lattice model.

\subsubsection{Local Functions}
N/A

\newpage

\section{MIS of M10: Boundary Module} \label{BOModule}

The secret of this module is the structure of the model boundary.

\subsection{Module}

Boundary

\subsection{Uses}
\begin{itemize}
	\item Data Structure (Section \ref{DSModule})
\end{itemize}

\subsection{Syntax}

\subsubsection{Exported Constants}
N/A
\subsubsection{Exported Access Programs}

\begin{center}
	\begin{tabular}{p{2cm} p{3cm} p{4cm} p{2cm}}
		\hline
		\textbf{Name} & \textbf{In} & \textbf{Out} & \textbf{Exceptions} \\
		\hline
		getBoundary & inputData: DataStructure & $\mathbb{R}$ & - \\
		\hline
	\end{tabular}
\end{center}

\subsection{Semantics}

\subsubsection{State Variables}

N/A

\subsubsection{Environment Variables}

N/A

\subsubsection{Assumptions}

The Problem Module M8 (Section \ref{PRModule}) has called Boundary.getBoundary().

\subsubsection{Access Routine Semantics}

\noindent getBoundary(inputData):
\begin{itemize}
	\item transition: $out$ := $\mathbb{R}$ $\in$ DataStructure
          \wss{DataStructure is not a set or a sequence (I don't think?), so
            this operation does not make sense.  Moreover, $\mathbb{R}$ is the
            set of real numbers.  For this to ever return True, DataStructure
            would have to be a set of sets of real numbers, with the possibility
            that one of those sets is all the real numbers.  Also, this
            expression evaluates to True of False, but the type of your output
            is real.}
	\item exception: N/A
\end{itemize}

The function returns the boundary from the DataStructure library.

\subsubsection{Local Functions}

N/A

\newpage

\section{MIS of M11: Image Rendering Module} \label{OUModule} 

The algorithm to convert the LBM output into an image.

\subsection{Module}

ImageRendering

\subsection{Uses}

N/A

\subsection{Syntax}

\subsubsection{Exported Constants}
N/A

\subsubsection{Exported Access Programs}

\begin{center}
	\begin{tabular}{p{2cm} p{4cm} p{4cm} p{2cm}}
		\hline
		\textbf{Name} & \textbf{In} & \textbf{Out} & \textbf{Exceptions} \\
		\hline
		imageFunc & imageData: array[$\mathbb{R}]$ & imageOut: PNG & IncorrectFormat \\
		\hline
	\end{tabular}
\end{center}

\subsection{Semantics}

\subsubsection{State Variables}
N/A

\subsubsection{Environment Variables}
N/A \wss{An environment variable for the png file would make sense here, rather
  than outputting the PNG file.}

\subsubsection{Assumptions}

The System Control Module M2 (Section \ref{PRModule}) has called ImageRendering.imageFunc() of this module.

\subsubsection{Access Routine Semantics}

\noindent imageFunc(imageData[]):
\begin{itemize}
	\item transition: $out$ := imageOut: PNG
	\item exception: ( imageData[] $\neq$ $\mathbb{R}$ $\Rightarrow$ IncorrectFormat)
\end{itemize}

The function converts the information from the LBM algorithm output (for example the vorticity vector values) into an image format using an external image library.

\subsubsection{Local Functions}
N/A

\newpage

\section{MIS of M12: Data Structure Module} \label{DSModule} 

The format of a non-primitive data structure for input variables.

\subsection{Module}

DataStructure

\subsection{Uses}
\begin{itemize}
	\item Input Types (Section \ref{ITModule})
\end{itemize}

\subsection{Syntax}

\subsubsection{Exported Constants}
N/A

\subsubsection{Exported Access Programs}

\begin{center}
	\begin{tabular}{p{2cm} p{4cm} p{4cm} p{2cm}}
		\hline
		\textbf{Name} & \textbf{In} & \textbf{Out} & \textbf{Exceptions} \\
		\hline
		init & - & - & - \\
		add & InputTypes, $\mathbb{R}$ & - & IncorrectFormat \\
		getElm &InputTypes & $\mathbb{R}$ & DoesNotExist \\
		\hline
	\end{tabular}
\end{center}

\subsection{Semantics}

\subsubsection{State Variables}

$s$: set of tuple of (key: InputType, value: $\mathbb{R}$)  \wss{You are using a
  variable as a type here.  Our mathematical framework does not handle dependent
  types.  I'm sure there is an abstract mathematics we could use here, but using
that would be a big jump when there is so much work to do to get the regular
Hoffman and Strooper math correct.  I suggest that you document for one LB type
of problem and drop the goal of being generic.}

\subsubsection{State Invariant}
N/A

\subsubsection{Assumptions}

DataStructure.init() is called before any other access program.

\subsubsection{Access Routine Semantics}

\noindent init():
\begin{itemize}
	\item transition: $s$ := \{\}
	\item exception: N/A
\end{itemize}

\noindent add(InputTypes, $\mathbb{R}$):
\begin{itemize}
	\item transition: $s$ := $s$ $\cup$ \{$\langle$ InputTypes, $\mathbb{R}$ $\rangle$\}
	\item exception: ($\langle$ InputTypes, value $\rangle$ $|$ value is not a $\mathbb{R}$ $\Rightarrow$ IncorrectFormat) 
\end{itemize}

\noindent remove(InputTypes):
\begin{itemize}
	\item transition: $s$ := $s$ - \{$\langle$ InputTypes, $\mathbb{R}$ $\rangle$\} where $\langle$ InputTypes, $\mathbb{R}$ $\rangle$ $\in $$s$
	\item exception: ($\langle$ InputTypes, $\mathbb{R}$ $\rangle$ $\notin$ $s$ $\Rightarrow$ DoesNotExist)
\end{itemize}

\noindent getElm(InputTypes):
\begin{itemize}
	\item transition: $out$ := $\mathbb{R}$ if $\langle$ InputTypes, $\mathbb{R}$ $\rangle$ $\in$ $s$
	\item exception: ($\langle$ InputTypes, $\mathbb{R}$ $\rangle$ $\notin$ $s$ $\Rightarrow$ DoesNotExist)
\end{itemize}

\subsubsection{Local Functions}
N/A

\newpage

\section{MIS of M13: Input Types Module} \label{ITModule} 

The input types for the non-primitive data structures.

\subsection{Module}

InputTypes

\subsection{Uses}

N/A

\subsection{Syntax}

\subsubsection{Exported Constants}
InputTypes (string) = \{Library, Problem, Dimensions, VelocityDirections,
ReynoldsNumber, Density, BulkViscosity, ShearViscosity, Time, Density,
Viscosity, AccelerationRate, SpeedSound, Velocity, Force, Mass,
CrossSectionalArea, MacroscopicVelocity, RelaxationRate\} \wss{You might be on
  to a good idea here, but I can't tell.  If there are a finite set of options,
  you should use an enumerated type, not a string.  I'm not sure what you are
  trying to do.  My advice above is to be less generic.  I think documenting one
  specific problem is enough of a challenge.}
 
\subsubsection{Exported Access Programs}
N/A

\subsection{Semantics}

\subsubsection{State Variables}
N/A

\subsubsection{State Invariant}
N/A

\subsubsection{Assumptions}

The Data Structure Module M12 (Section \ref{DSModule}) will only use a set of the above InputTypes.

\subsubsection{Access Routine Semantics}
N/A

\subsubsection{Local Functions}
N/A

\newpage

\bibliographystyle {plainnat}
\bibliography{../../../refs/References}

\newpage

\section{Appendix} \label{Appendix}

\begin{longtable}{l p{12cm}}
	\caption{Possible Exceptions} \\
	\toprule
	\textbf{Message ID} & \textbf{Error Message} \\
	\midrule
	DoesNotExist & Error: The input type does not exist.\\ 
	ErrorRead & Error: Could not read the input file. \\
	IncorrectFormat & Error: The format is not a real number.\\
	NAN & Error: The calculated result is not a number.\\
	NoLattice & Error: The chosen velocity directions do not have a known lattice structure.\\
	NotFound & Error: Input file not found. \\
	OutBounds & Error: The input file parameter X is out of bounds. Please see the User Guide.\\
	UnknwnParm & Error: The parameter X is not known to the system. Please see the User Guide.\\	
	\bottomrule
\end{longtable}


\end{document}