\documentclass{article}
\makeatletter
\renewcommand{\@seccntformat}[1]{}
\makeatother
\usepackage{tabularx}
\usepackage{booktabs}
\usepackage[margin=1.64in]{geometry}
\title{CAS 741: Problem Statement\\ Commonality, Variability, and Implementation of Lattice Boltzmann Solvers}

\author{Peter Michalski \\* 000653483}

\date{2018-12-16}

%% Comments

\usepackage{color}

\newif\ifcomments\commentstrue

\ifcomments
\newcommand{\authornote}[3]{\textcolor{#1}{[#3 ---#2]}}
\newcommand{\todo}[1]{\textcolor{red}{[TODO: #1]}}
\else
\newcommand{\authornote}[3]{}
\newcommand{\todo}[1]{}
\fi

\newcommand{\wss}[1]{\authornote{blue}{SS}{#1}} 
\newcommand{\plt}[1]{\authornote{magenta}{TPLT}{#1}} %For explanation of the template
\newcommand{\an}[1]{\authornote{cyan}{Author}{#1}}


\begin{document}

\maketitle

\begin{table}[hp]
\caption{Revision History} \label{TblRevisionHistory}
\begin{tabularx}{\textwidth}{llX}
\toprule
\textbf{Date} & \textbf{Developer(s)} & \textbf{Change}\\
\midrule
2018-12-17 & P. Michalski & Initial Draft\\
\bottomrule
\end{tabularx}
\end{table}

\section{Problem}
The Lattice Boltzmann methods (LBM) are a powerful technique to simulate multiphase and multicomponent fluid dynamics using a mesoscopic distribution function. The LBM have grown in popularity since its inception and there are now several open-source implementations of LBM for simulating fluid dynamics over a range of parameters. These implementations are of varying complexity and they offer many distinct parameter options, which can be challenging for the user.
   
\section{Solution}
This project will conduct a commonality analysis for the aforementioned family of LBM solvers, and will attempt to distinguish key solver functionality. A new solver will be built that will include the most important features found in the commonality analysis, providing the user with a basic, easy to use, implementation of LBM.

\section{Context}
\subsection{Environment}
The solver will be compatible with KDE neon 5.16 (Ubuntu 18.04 LTS) and macOS 10.13.6. Compatibility with other operating systems will not be guaranteed.

\subsection{Stakeholders}
Stakeholders include:
\begin{itemize}
\item Dr. Spencer Smith
\item Dr. Jacques Carette
\item Ao Dong
\item Other members of my M.Eng project team
\item Individuals studying or working with Lattice Boltzmann implementations
\end{itemize}

\end{document}