\documentclass[12pt]{article}

\usepackage{amsmath, mathtools}
\usepackage{amsfonts}
\usepackage{amssymb}
\usepackage{graphicx}
\usepackage{colortbl}
\usepackage{xr}
\usepackage{hyperref}
\usepackage{longtable}
\usepackage{xfrac}
\usepackage{tabularx}
\usepackage{float}
\usepackage{siunitx}
\usepackage{booktabs}
\usepackage{caption}
\usepackage{pdflscape}
\usepackage{afterpage}
\usepackage{caption}
\usepackage{pbox}
\usepackage{makecell}




\usepackage[sort&compress,square,comma,numbers]{natbib}

\DeclareRobustCommand{\citeext}[1]{\citeauthor{#1}~\cite{#1}}

%\usepackage{refcheck}

\hypersetup{
    bookmarks=true,         % show bookmarks bar?
      colorlinks=true,       % false: boxed links; true: colored links
    linkcolor=red,          % color of internal links (change box color with linkbordercolor)
    citecolor=blue,        % color of links to bibliography
    filecolor=magenta,      % color of file links
    urlcolor=cyan           % color of external links
}

%% Comments

\usepackage{color}

\newif\ifcomments\commentstrue

\ifcomments
\newcommand{\authornote}[3]{\textcolor{#1}{[#3 ---#2]}}
\newcommand{\todo}[1]{\textcolor{red}{[TODO: #1]}}
\else
\newcommand{\authornote}[3]{}
\newcommand{\todo}[1]{}
\fi

\newcommand{\wss}[1]{\authornote{blue}{SS}{#1}} 
\newcommand{\plt}[1]{\authornote{magenta}{TPLT}{#1}} %For explanation of the template
\newcommand{\an}[1]{\authornote{cyan}{Author}{#1}}


% For easy change of table widths
\newcommand{\colZwidth}{1.0\textwidth}
\newcommand{\colAwidth}{0.13\textwidth}
\newcommand{\colBwidth}{0.82\textwidth}
\newcommand{\colCwidth}{0.1\textwidth}
\newcommand{\colDwidth}{0.05\textwidth}
\newcommand{\colEwidth}{0.8\textwidth}
\newcommand{\colFwidth}{0.17\textwidth}
\newcommand{\colGwidth}{0.5\textwidth}
\newcommand{\colHwidth}{0.28\textwidth}

% Used so that cross-references have a meaningful prefix
\newcounter{defnum} %Definition Number
\newcommand{\dthedefnum}{GD\thedefnum}
\newcommand{\dref}[1]{GD\ref{#1}}
\newcounter{datadefnum} %Datadefinition Number
\newcommand{\ddthedatadefnum}{DD\thedatadefnum}
\newcommand{\ddref}[1]{DD\ref{#1}}
\newcounter{theorynum} %Theory Number
\newcommand{\tthetheorynum}{T\thetheorynum}
\newcommand{\tref}[1]{T\ref{#1}}
\newcounter{tablenum} %Table Number
\newcommand{\tbthetablenum}{T\thetablenum}
\newcommand{\tbref}[1]{TB\ref{#1}}
\newcounter{assumpnum} %Assumption Number
\newcommand{\atheassumpnum}{P\theassumpnum}
\newcommand{\aref}[1]{A\ref{#1}}
\newcounter{goalnum} %Goal Number
\newcommand{\gthegoalnum}{P\thegoalnum}
\newcommand{\gsref}[1]{GS\ref{#1}}
\newcounter{instnum} %Instance Number
\newcommand{\itheinstnum}{IM\theinstnum}
\newcommand{\iref}[1]{IM\ref{#1}}
\newcounter{reqnum} %Requirement Number
\newcommand{\rthereqnum}{P\thereqnum}
\newcommand{\rref}[1]{R\ref{#1}}
\newcounter{lcnum} %Likely change number
\newcommand{\lthelcnum}{LC\thelcnum}
\newcommand{\lcref}[1]{LC\ref{#1}}

\newcommand{\famname}{Lattice Boltzmann Solvers} % PUT YOUR PROGRAM NAME HERE

\usepackage{fullpage}

\begin{document}

\title{\famname} 
\author{Peter Michalski}
\date{\today}

\maketitle

~\newpage

\pagenumbering{roman}

\section{Revision History}

\begin{tabularx}{\textwidth}{p{3cm}p{2cm}X}
\toprule {\bf Date} & {\bf Version} & {\bf Notes}\\
\midrule
October 7,2019 & 1.0 & Initial Document\\
\bottomrule
\end{tabularx}

~\newpage
	
\section{Reference Material}

This section records information for easy reference.

\subsection{Table of Units}

Throughout this document SI (Syst\`{e}me International d'Unit\'{e}s) is employed
as the unit system.  In addition to the basic units, several derived units are
used as described below.  For each unit, the symbol is given followed by a
description of the unit and the SI name.
~\newline

\renewcommand{\arraystretch}{1.2}
%\begin{table}[ht]
  \noindent \begin{tabular}{l l l} 
    \toprule		
    \textbf{symbol} & \textbf{unit} & \textbf{SI}\\
    \midrule 
    \si{\metre} & length & metre\\
    \si{\kilogram} & mass	& kilogram\\
    $t$ & time & second\\
    $F$ & force & newton\\
    $cm$ & length & centimetre\\
    $g$ & mass & gram \\
    \bottomrule
  \end{tabular}
  %	\caption{Provide a caption}
%\end{table}

\subsection{Table of Symbols}

The table that follows summarizes the symbols used in this document along with
their units.  The choice of symbols was made to be consistent with the heat
transfer literature and with existing documentation for solar water heating
systems.  The symbols are listed in alphabetical order.

\renewcommand{\arraystretch}{1.2}
%\noindent \begin{tabularx}{1.0\textwidth}{l l X}
\noindent \begin{longtable*}{l l p{12cm}} \toprule
\textbf{symbol} & \textbf{unit} & \textbf{description}\\
\midrule 
$e$ & $\frac{m}{s}$ & velocity
\\
$\eta$ & $Pa-s$ & viscosity
\\ 
$A$ & $m^2$ & cross-sectional area
\\
$\gamma$ & $\frac{1}{s}$ & velocity gradient
\\
$\tau$ & N/A & relaxation rate
\\
$x$ & N/A & position vector
\\
$f$ & N/A & distribution function
\\
$\Omega$ & N/A & collision operator
\\
$f^{eq}$ & N/A & equilibrium distribution function
\\
$k$ & N/A & velocity direction
\\
$p$ & $\frac{g}{cm^3}$ & fluid density
\\
$w$ & N/A & weight coefficient (implementation specific)
\\
$u$ & $\frac{m}{s}$ & macroscopic velocity of fluid
\\
$\mathrm{D}$ & N/A & signifies the dimension component of lattice model
\\
$\mathrm{Q}$ & N/A & signifies number of velocity directions of lattice model
\\
$\sigma$ & N/A & variable number of dimensions in the lattice model
\\
$\kappa$ & N/A & variable number of velocity directions of lattice model (linkages)
\\
$\mathbb{R}$ & N/A & real numbers
\\
$c_k$ & NA & unit vector along the lattice streaming direction
\\
$c_s$ & $\frac{m}{s}$ & speed of sound
\\
\bottomrule
\end{longtable*}

\subsection{Abbreviations and Acronyms}

\renewcommand{\arraystretch}{1.2}
\begin{tabular}{l l} 
  \toprule		
  \textbf{symbol} & \textbf{description}\\
  \midrule
  1D & 1-Dimensional\\ 
  2D & 2-Dimensional\\ 
  3D & 3-Dimensional\\ 
  A & Assumption\\
  CA & Commonality Analysis\\
  DD & Data Definition\\
  GS & Goal Statement\\
  LBM & Lattice Boltzmann Methods\\
  LBS & Lattice Boltzmann Solvers\\
  LC & Likely Change\\
  MPI & Message Passing Interface\\
  OTS & Off The Shelf (LBS Solutions)\\
  R & Requirement\\
  T & Theoretical Model\\
  \bottomrule
\end{tabular}\\

\newpage

\tableofcontents

\pagenumbering{arabic}

\section{Introduction}

This document provides a Commonality Analysis (CA) for a family of Lattice Boltzmann Solvers (LBS), which provide services based on Lattice Boltzmann Methods (LBM).
LBM are a family of fluid dynamics algorithms for simulating single-phase and multiphase fluid flows, often incorporating additional physical complexities \citet{chen1998lattice}. They consider the behaviours of a collection of particles as a single unit at the mesoscopic scale. These methods predict the positional probability of a collection of particles moving through a lattice structure. Off the shelf (OTS) Lattice Boltzmann Solvers (LBS) allow for a range of fluid and physical model input parameters, computational parameters, and output parameters as outlined in Section \ref{OTSsolutions}.
The following subsection of this introduction will outline the purpose of this document, a general scope of the family of LBS, the characteristics of the intended reader, and finally an outline of the rest of this document.

\subsection{Purpose of Document}

The purpose of this document is to provide general information on the currently available LBS solutions, including their commonalities and variabilities, as well as a baseline understanding of the model and structure of abstract LBM. The information provided here will be used in the development of the design of a solution providing services of a family of LBS.

\subsection{Scope of the Family} 

The family of LBS will model one or more fluids as they pass through a boundary, modeled by a lattice. Fluids with any properties can be modeled, however only those properties that are accepted as inputs by the LBS will affect the model results. The calculation of the LBM distribution function will use up to 3D computational models, and will output the data into memory and render it in up to 3D imaging.

\subsection{Characteristics of Intended Reader} 

The intended reader of this document should have an undergraduate understanding of software requirements and specifications as well as software design principles. Ideally, the user will be knowledgeable of commonality analysis, 

\subsection{Organization of Document}

This document is organized along a template for a CA for scientific computing software proposed by \citet{smith2006systematic}. It follows a standard pattern of presenting a general system description, commonalities, variabilities, and the requirements for a family of LBS. The goal statements of the family of LBS, found in Section \ref{goalstatements}, are refined to the theoretical models in Section \ref{sec_theoretical}. Variabilities within the family are found in Section \ref{variabilities}. Tables of OTS solution commonalities and variabilities are found in Section \ref{OTSsolutions}.

\section{General System Description}

This section identifies the interfaces between the system and its environment,
describes the potential user characteristics and lists the potential system
constraints.

\subsection{Potential System Contexts}

\begin{itemize}
\item User Responsibilities:
\begin{itemize}
\item The user must provide the system with correctly formatted physical model parameters.
\item The user must select the desired mathematical model for the computation.
\item The user must select the desired format of output for the model.
\end{itemize}
\item \famname{} Responsibilities:
\begin{itemize}
\item Detect data type mismatch, such as a negative number instead of a positive number for a parameter, such as $A$ that cannot accept negative values.
\item Initialize the correct data types and data structures for the model.
\item Perform the calculations to predict the distribution of fluid particles over time.
\item Store the distribution function output data.
\item Store calculated fluid parameters over time.
\item Visually model the results of the distribution function.
\item Store the calculation results in a file and/or in memory.
\item Detect errors during parameter input, model calculation, or model output; store the errors in a file and show the error to the user.
\item Recover from error states, such as those that develop from division by zero or a buffer overflow.
\end{itemize}
\end{itemize}

\subsection{Potential User Characteristics} \label{SecUserCharacteristics}

The end user of \famname{} should ideally have an understanding of undergraduate Level 1 physics and fluid dynamics. The ideal end user characteristics may differ between the members of the family of solvers. For example, a user of HemeLB, a off the shelf LBM solution for simulating blood flow, would ideally have an understanding of phlebology.

\subsection{Potential System Constraints}
\label{systemconstraints}

The parallel nature of LBS prefers operating and hardware systems that can handle concurrency and large amounts of data. Modern operating systems and computer hardware platforms are suggested. Memory should be scaled to the requirements of the desired LBS library, and decomposition technique.

\section{Commonalities}

\subsection{Background Overview} \label{Sec_Background}

As LBS model fluid dynamics within a boundary using a predefined lattice structure, the methods rely on a two step calculation process. The first processes is known as streaming, where the particles move along the lattice via links, and the second process is collision, where energy and momentum is transferred among particles that collide \cite{bao2011lattice}.
In the LBS solutions, the particles are mapped using a lattice structure. The lattice structure can be a 1D, 2D, or 3D model with varying velocity directions. The notation is D$\sigma$Q$\kappa$, where $\sigma$ represents the number of dimensions and $\kappa$ represents the number of velocity directions.There are many standardized lattice models; individual solvers within the family may only use a subset of them.
The LBM uses the initial parameters of the fluid to find the probability of where along the lattice linkages a group of particles are most likely to travel. It then moves the particles into the next node, and transfers the energy and momentum if a collision occurs. Then the process repeats for the duration of the modeling instance.

\subsection{Terminology and  Definitions}
\label{termdef}

This subsection provides a list of terms that are used in the subsequent
sections and their meaning, with the purpose of reducing ambiguity and making it
easier to correctly understand the requirements:

\begin{itemize}

\item Correctness : The degree to which a system or component is free from faults in its specification, design, and implementation \citet{IEEEStdGlossarySET1990}.
\item Maintainability: The ease with which a software system or component can be modified to correct faults, improve performance or other attributes, or adapt to a changed environment \citet{IEEEStdGlossarySET1990}.
\item Performance: The degree to which a system or component accomplishes its designated functions within given constraints, such as speed, accuracy, or memory usage \citet{IEEEStdGlossarySET1990}.
\item Portability: The ease with which a system or component can e transferred from one hardware or software environment to another \citet{IEEEStdGlossarySET1990}.
\item Reliability: The ability of a system or component to perform its required functions under stated conditions for a specified period of time \citet{IEEEStdGlossarySET1990}.
\item Reusability: The degree to which a software module or other work product can be used in more than one computer program or software system \citet{IEEEStdGlossarySET1990}.
\item Robustness: The degree to which a system or component can function correctly in the presence of invalid inputs or stressful environmental conditions \citet{IEEEStdGlossarySET1990}.
\item Scalability: The ability of the system to cope with increasing numbers of users without reducing overall QoS that is delivered to any user \citet{sommerville}.
\item Understandability: The ease of understanding the software system  \citet{uchida2005experiment}.
\item Usability: The ease with which a user can learn to operate, prepare inputs for, and interpret outputs of a system or component \citet{IEEEStdGlossarySET1990}.

\end{itemize}

\subsection{Data Definitions} \label{sec_datadef}

This section collects and defines all the data needed to build the instance models. The dimension of each quantity is also given.  

~\newline

\noindent
\begin{minipage}{\textwidth}
\renewcommand*{\arraystretch}{1.5}
\begin{tabular}{| p{\colAwidth} | p{\colBwidth}|}
\hline
\rowcolor[gray]{0.9}
Number& DD\refstepcounter{datadefnum}\thedatadefnum \label{DD_Velocity}\\
\hline
Label& \bf Velocity\\
\hline
Symbol &$\mathrm{e}$\\
\hline
% \hline
  SI Units & \si{\frac\metre\second}\\
  \hline
  Equation& $\mathrm{e} = \frac{d \mathrm{r}}{dt}$\\
  \hline
  Description & 
                 Velocity is the distance that an object moves relative to time. $r$ is the the distance ($\mathrm{m}$) in change for our change in time {t} of units ($\mathrm{s}$).
  \\
  \hline
  Sources& \citet{mohamad2011lattice}\\
  \hline
  Ref.\ By & \tref{T_BTE} \tref{T_EDF}\\
  \hline
\end{tabular}
\end{minipage}\\


~\newline


\noindent
\begin{minipage}{\textwidth}
\renewcommand*{\arraystretch}{1.5}
\begin{tabular}{| p{\colAwidth} | p{\colBwidth}|}
\hline
\rowcolor[gray]{0.9}
Number& DD\refstepcounter{datadefnum}\thedatadefnum \label{DD_Viscosity}\\
\hline
Label& \bf Viscosity\\
\hline
Symbol &$\mathrm{\eta}$\\
\hline
% \hline
  SI Units & $\mathrm{Pa-s}$\\
  \hline
  Equation& $\mathrm{\eta} = \frac{F/A}{\gamma}$\\
  \hline
  Description & 
                Viscosity is the measure of resistance to deformation. $F$ is the applied force (N), $A$ is the cross-sectional area ($m^2$), and $\gamma$ is the velocity gradient. 
  \\
  \hline
  Sources& \citet{viscosity}\\
  \hline
  Ref.\ By & DD\ref{DD_RelaxationRate}\\
  \hline
\end{tabular}
\end{minipage}\\

~\newline

\noindent
\begin{minipage}{\textwidth}
\renewcommand*{\arraystretch}{1.5}
\begin{tabular}{| p{\colAwidth} | p{\colBwidth}|}
\hline
\rowcolor[gray]{0.9}
Number& DD\refstepcounter{datadefnum}\thedatadefnum \label{DD_RelaxationRate}\\
\hline
Label& \bf Relaxation Rate Towards Equilibrium\\
\hline
Symbol &$\tau$\\
\hline
% \hline
  SI Units & NA\\
  \hline
  Equation&$\tau = \frac{12\mathrm{\eta}\Delta t}{\Delta\mathrm{x}^2} + \frac{1}{2}$\\
  \hline
  Description & 
                The relaxation rate defines how quickly the particles recover to equilibrium state. Adjusting this method in the implementation allows for the simulation of complex physical phenomena, specifically concerning the fluid media. $\mathrm{\eta}$ is the viscosity of the fluid, $t$ is the time interval (s), and $x$ is the position vector.
  \\
  \hline
  Sources& \citet{lbmbolton}\\
  \hline
  Ref.\ By & \tref{T_BTE}\\
  \hline
\end{tabular}
\end{minipage}\\

~\newline

\noindent
\begin{minipage}{\textwidth}
\renewcommand*{\arraystretch}{1.5}
\begin{tabular}{| p{\colAwidth} | p{\colBwidth}|}
\hline
\rowcolor[gray]{0.9}
Number& DD\refstepcounter{datadefnum}\thedatadefnum \label{DD_VelocityGradient}\\
\hline
Label& \bf Velocity Gradient\\
\hline
Symbol &$\gamma$\\
\hline
% \hline
  SI Units &$\frac{1}{s}$\\
  \hline
  Equation&$\gamma = \frac{d\mathrm{e}}{dz}$\\
  \hline
  Description & 
                Velocity gradient is the difference in velocity between adjacent fluids. $d\mathrm{e}$ represents the difference in velocities of the fluids and $dz$ is the
distance of the two velocities.  \\
  \hline
  Sources& \citet{viscosity}\\
  \hline
  Ref.\ By & DD\ref{DD_Viscosity}\\
  \hline
\end{tabular}
\end{minipage}\\

~\newline

\noindent
\begin{minipage}{\textwidth}
\renewcommand*{\arraystretch}{1.5}
\begin{tabular}{| p{\colAwidth} | p{\colBwidth}|}
\hline
\rowcolor[gray]{0.9}
Number& DD\refstepcounter{datadefnum}\thedatadefnum \label{DD_FluidDensity}\\
\hline
Label& \bf Fluid Density\\
\hline
Symbol &$p$\\
\hline
% \hline
  SI Units &$\frac{g}{cm^3}$ \\
  \hline
  Equation& $p$ = $\frac{g}{cm^3}$ \\
  \hline
  Description & 
                Density is the ratio of mass to volume of a material. $g$ is the mass and $cm^3$ is the volume.  \\
  \hline
  Sources& \citet{density}\\
  \hline
  Ref.\ By & \tref{T_EDF}\\
  \hline
\end{tabular}
\end{minipage}\\


\subsection{Goal Statements}
\label{goalstatements}
\noindent Given the boundary conditions, lattice model, weighting coefficient of the lattice, simulation time, fluid particle mass, and initial conditions for the momentum, density and position of the fluid particles, as well as any applied external force, the goal statements are:

\begin{itemize}

\item[\label{G_Location}]G\_prob: Predict the location of fluid particles in the lattice over time.

\item[\label{G_Velocity}]G\_velocity: Predict the velocity of fluid particles within the lattice over time.

\item[\label{G_FluidPressure}]G\_fluidPressure: Predict the pressure of fluid particles within the lattice over time.

\end{itemize}

~\newpage

\subsection{Theoretical Models} \label{sec_theoretical}

This section focuses on the general equations and laws that \famname{} are based
on.  

~\newline

\noindent
\begin{minipage}{\textwidth}
\renewcommand*{\arraystretch}{1.5}
\begin{tabular}{| p{\colAwidth} | p{\colBwidth}|}
  \hline
  \rowcolor[gray]{0.9}
  Number& T\refstepcounter{theorynum}\thetheorynum \label{T_BTE}\\
  \hline
  Label&\bf Boltzmann Transport Equation\\
  \hline
  Equation&  $f(\mathrm{x} +\mathrm{e}dt, \mathrm{e} + \frac{\mathrm{F}}{\mathrm{m}}dt, t + dt)d\mathrm{x}d\mathrm{e} - f(\mathrm{x},\mathrm{e},t)d\mathrm{x}d\mathrm{e} = \mathrm{\Omega}(f)d\mathrm{x}d\mathrm{e}$\\
  \hline
  Description & 
  This equation determines the statistical description of a group of particles. The left part of the equation, $f(\mathrm{x} +\mathrm{e}dt, \mathrm{e} + \frac{\mathrm{F}}{\mathrm{kg}}dt, t + dt)d\mathrm{x}d\mathrm{e}$, represents the distribution function result after an external force $\mathrm{F}$ is applied. The middle function, $f(\mathrm{x},\mathrm{e},t)d\mathrm{x}d\mathrm{e}  $, represents the distribution function result before the external force is applied. The distribution function $f$ represents the probability that a set of particles will be at a specific location of the lattice at a specified time. The right hand side of the equation represents the collision operator, $\Omega$.\newline
The variable $\mathrm{x}$ represents the vector of the particles within the lattice, $\mathrm{e}$ is velocity $\mathrm{\frac{m}{s}}$, $\mathrm{t}$ is time ($\mathrm{s}$), $\mathrm{F}$ is force ($\mathrm{N}$), $\mathrm{kg}$ is length ($\mathrm{kg}$). This equation can be further developed for specific instances. The user shall select the desired model as stated in A\ref{A_selectModel}, and input the desired starting parameters as stated in A\ref{A_userInputs}.\\
  \hline
  Source &
           \citet{lbmbolton}\newline \citet{mohamad2011lattice}\\
  % The above web link should be replaced with a proper citation to a publication
  \hline
  Ref.\ By & \\
  \hline
\end{tabular}
\end{minipage}\\

~\newline

\noindent
\begin{minipage}{\textwidth}
\renewcommand*{\arraystretch}{1.5}
\begin{tabular}{| p{\colAwidth} | p{\colBwidth}|}
  \hline
  \rowcolor[gray]{0.9}
  Number& T\refstepcounter{theorynum}\thetheorynum \label{T_BGK}\\
  \hline
  Label&\bf Bhatnagar-Gross-Krook Collision Operator\\
  \hline
  Equation&  $\mathrm{\Omega} = \frac{\Delta t}{\tau}(f^{eq}(r,t)-f(r,t))$\\
  \hline
  Description &
  The above equation is a mathematical operator that preserves continuity for a discretized model.
  $\tau$ is the relaxation rate towards equilibrium and should be in the range of 0.5 - 2.0. It is related to viscosity as outlined in DD. 
  $f^{eq}$ is the equilibrium particle probability distribution function. $f$ is the particle probability distribution function. This equation can be further developed for specific instances. Several fluids in the instance model can be modeled by this equation, as stated in A\ref{A_fluids}. The user shall select the desired model as stated in A\ref{A_selectModel}, and input the desired starting parameters as stated in A\ref{A_userInputs}.\\
  \hline
  Source &
           \citet{lbmbolton}\newline \citet{mohamad2011lattice}\\
  \hline
  Ref.\ By & \tref{T_BTE}\\
  \hline
\end{tabular}
\end{minipage}\\

~\newline

\noindent
\begin{minipage}{\textwidth}
\renewcommand*{\arraystretch}{1.5}
\begin{tabular}{| p{\colAwidth} | p{\colBwidth}|}
  \hline
  \rowcolor[gray]{0.9}
  Number& T\refstepcounter{theorynum}\thetheorynum \label{T_EDF}\\
  \hline
  Label&\bf Equilibrium Distribution Function\\
  \hline
  Equation&  $f_{k}^{eq} = pw_{k}[1 + \frac{2\overrightarrow{c_k}\overrightarrow{u}-\overrightarrow{u}\overrightarrow{u}}{2c_s^2}+\frac{(\overrightarrow{c_k}\overrightarrow{u})^2}{2c_s^4}] + O(u^2)$\\
  \hline
  Description &
  The above equation captures the probability distribution of the particles. Adjusting this method in the implementation allows for the simulation of complex physical phenomena, including geometry of the boundary.
  $p$ is the fluid density $(\mathrm{\frac{g}{cm^2}}$). $w$ is the weighting coefficient for the lattice model as the fluid flows through a lattice structure A\ref{A_lattice}. The weighting coefficients are standard, as per A\ref{A_weightCoefficients}. $k$ is the discretized velocity direction, referring to the directions of the chosen lattice model. $\mathrm{c_k}$ is the unit vector along the lattice streaming direction. $\mathrm{u}$ is the macroscopic velocity of the fluid, which is a vector field of velocity at a specific position and time. $\mathrm{c_s}$ is the speed of sound, a constant, as stated in A\ref{A_speedSound}. This equation can be further developed for specific instances. Several fluids in the instance model can be modeled by this equation, as stated in A\ref{A_fluids}. Several fluids in the instance model can be modeled by this equation, as stated in A\ref{A_flowObject}, can be reflected in the velocity direction weights. The user shall select the desired model as stated in A\ref{A_selectModel}, and input the desired starting parameters as stated in A\ref{A_userInputs}.\\
  \hline
  Source &
           \citet{lbmbolton}
           \newline \citet{mohamad2011lattice}
           \\
  % The above web link should be replaced with a proper citation to a publication
  \hline
  Ref.\ By & \tref{T_BGK}\\
  \hline
\end{tabular}
\end{minipage}\\

~\newline
~\newpage

\section{Variabilities}
\label{variabilities}

\subsection{Assumptions}

\begin{itemize}

\item[A\refstepcounter{assumpnum}\theassumpnum \label{A_fluids}:]
  One or more fluids can be modeled.
  
\item[A\refstepcounter{assumpnum}\theassumpnum \label{A_flowObject}:]
  The fluid can, but does not need to, flow through an object with boundary conditions.
  
\item[A\refstepcounter{assumpnum}\theassumpnum \label{A_lattice}:]
  The fluid flows through space via a lattice structure, moving between lattice nodes via linkages ($Q$).
  
\item[A\refstepcounter{assumpnum}\theassumpnum \label{A_weightCoefficients}:]
  Weight coefficients are standard for each lattice model. See the Table \ref{coefficientweights}.

\item[A\refstepcounter{assumpnum}\theassumpnum \label{A_selectModel}:]
  The user will select the desired model prior to running the simulation.

\item[A\refstepcounter{assumpnum}\theassumpnum \label{A_userInputs}:]
  The user will enter a subset of fluid parameter inputs prior to running the simulation. See Table \ref{table:inputVar}.   
  
\item[A\refstepcounter{assumpnum}\theassumpnum \label{A_speedSound}:]
  The speed of sound is constant. It's value can be found in Section \ref{symbolicpara}. The constant is referenced in model \tref{T_EDF}.   

\end{itemize}

\begin{table}[!h]
\begin{center}
\begin{tabular}{| c | c | c | c | c | c | c | c | c |}
\hline
\textbf{Goal} & \textbf{Model} & \textbf{A1} & \textbf{A2} & \textbf{A3} & \textbf{A4} & \textbf{A5} & \textbf{A6} & \textbf{A7} \\
\hline
G1 & T1 & & & & & \checkmark & \checkmark &\\
\hline
G1 & T2 & \checkmark & & & & \checkmark & \checkmark &\\
\hline
G1 & T3 & \checkmark & \checkmark & \checkmark & \checkmark & \checkmark & \checkmark & \checkmark \\
\hline
G2 & T1 & & & & & \checkmark & \checkmark &\\
\hline
G2 & T2 & \checkmark & & & & \checkmark & \checkmark & \\
\hline
G2 & T3 & \checkmark & \checkmark & \checkmark & \checkmark & \checkmark & \checkmark & \checkmark \\
\hline
G3 & T1 & & & & & \checkmark & \checkmark & \\
\hline
G3 & T2 & \checkmark & & & & \checkmark & \checkmark &\\
\hline
G3 & T3 & \checkmark & & & & \checkmark & \checkmark & \checkmark \\
\hline
\end{tabular}
\caption{Assumption Relationship to Goals and Models}
\end{center}
\end{table}   

~\newpage

\subsection{Calculation} \label{sec_Calculation}

\begin{table}[!h]
\begin{center}
\begin{tabular}{| c | c | c |}
\hline
\textbf{Variability} & \textbf{Parameter of Variation} & \textbf{Binding Time}\\
\hline
boundary shape & Set of \{defined 2D, defined 3D, undefined\} & compile \\
\hline
boundary parameters & Set of \{deflective, non deflective\} & compile \\
\hline
fluid parameters & Set of \{$e$, $t$, $u$, $p$, $x$, $\eta$, $\tau$, $\gamma$, $F$, $A$\} & compile \\
\hline
model choice & Set of \{1D, 2D, 3D\} & compile \\
\hline
velocity directions & Set of \{2, 3, 5, 9,  13, 15, 19, 27\} & scope \\
\hline
velocity ($e$) & Set of \mathbb{R} & scope \\
\hline
time ($t$) & Set of \mathbb{R}& scope \\
\hline
macroscopic velocity ($u$) & Set of \mathbb{R}& scope \\
\hline
fluid density ($p$) & Set of positive \mathbb{R}& scope \\
\hline
position vector ($x$) & Vector of Set of \mathbb{R}& scope \\
\hline
viscosity ($\eta$) & Set of \mathbb{R} $\geq$ 0 & scope \\
\hline
relaxation rate ($\tau$) & Set of \mathbb{R} & scope \\
\hline
velocity gradient ($\gamma$ )& Set of \mathbb{R} & scope\\
\hline
force ($F$) & Set of \mathbb{R} $\geq$ 0 & scope \\
\hline
cross-sectional area ($A$) & Set of \mathbb{R} $\geq$ 0 & scope \\
\hline
\end{tabular}
\caption{Input Variabilities}
\label{table:inputVar}
\end{center}
\end{table}

\begin{table}[!h]
\begin{center}
\begin{tabular}{| c | c | c |}
\hline
\textbf{Variability} & \textbf{Parameter of Variation} & \textbf{Binding Time}\\
\hline
\pbox{4cm}{computational model (see Section \ref{systemconstraints}) }& \pbox{5cm}{ \\ D1Q2, D1Q3, D1Q5, D2Q9, D2Q13, D2Q15, D3Q15, D3Q15i, D3Q19, D3Q19+, D3Q27} & compile \\
\hline
\pbox{4.75cm}{decomposition technique (see Section \ref{systemconstraints}) }& Set of \pbox{6cm}{\{ParMETIS library, PT\textunderscore Scotch library, block-wide decomposition, domain decomposition, spinoidal decomposition\}} & compile \\
\hline
coefficient weights & Set of \{0 $<$ \mathbb{R} $<$ 1\}; \sum \mathbb{R} = 1 & compile \\
\hline
input check & \pbox{6cm}{boolean (false if input satisfies input assumptions)} & compile \\
\hline
exception check & \pbox{6cm}{boolean (false if no exception condition raised)} & compile \\
\hline
\end{tabular}
\caption{Calculation Variabilities}
\label{calcVar}
\end{center}
\end{table}

~\newpage

\subsection{Output} \label{sec_Output} 

\begin{table}[!h]
\begin{center}
\begin{tabular}{| c | c | c |}
\hline
Variability & Parameter of Variation & Binding Time\\
\hline
graphical model & Set of \{2D, 3D\}; \sum \mathbb{R} = 1 & run \\
\hline
destination for output & Set of \{file, screen, memory\} = 1 & run \\
\hline
fluid characteristics & Set of \pbox{5cm}{\{wall pressure, flow velocity, fluid location\} }& run \\
\hline
encoding of output & Set of \{binary, text\} & run \\
\hline
\end{tabular}
\caption{Output Variabilities}
\end{center}
\end{table}   

~\newpage

\section{Requirements}

This section provides the functional requirements, the business tasks that the
software is expected to complete, and the nonfunctional requirements, the
qualities that the software is expected to exhibit.

\subsection{Family of Functional Requirements}

\noindent \begin{itemize}

\item[R\refstepcounter{reqnum}\thereqnum \label{R_Inputs}:] The user shall input a set of fluid parameters, listed in Table \ref{table:inputVar}, into the system, as per A\ref{A_userInputs}. These parameters will be used in calculations for \tref{T_BTE}, \tref{T_BGK}, and \tref{T_EDF}.

\item[R\refstepcounter{reqnum}\thereqnum \label{R_ModelInputs}:] The user shall select from a set of model and velocity direction parameters, listed in Table \ref{table:inputVar}, into the system, as per A\ref{A_selectModel}. These models will be reflected in the calculations of \tref{T_BTE}.

\item[R\refstepcounter{reqnum}\thereqnum \label{R_CheckInputs}:] The system shall verify that the inputs fall within the allowable parameters of variation, see Table \ref{table:inputVar}.

\item[R\refstepcounter{reqnum}\thereqnum \label{R_Instantiate}:] The system shall instantiate required data types and structures for the selected model.

\item[R\refstepcounter{reqnum}\thereqnum \label{R_CoefficientWeights}:] The system shall import from memory the relevant coefficient weights for the selected model, as per A\ref{A_weightCoefficients}. The weighting values can be found in Table \ref{coefficientweights}. These models will be reflected in the calculations of \tref{T_BTE}.

\item[R\refstepcounter{reqnum}\thereqnum \label{R_Calculate}:] The system shall calculate and store in memory the predicted fluid parameters, iterating through streaming and collision processes over the desired model time.

\item[R\refstepcounter{reqnum}\thereqnum \label{R_Output}:] The system shall output the results of the calculations to the screen, to a file, and/or to memory.

\end{itemize}

~\newpage

\subsection{Nonfunctional Requirements}

The following are non-functional requirements for the family of LBS. They are defined in Section \ref{termdef}:

\begin{enumerate}
\item Correctness
\item Maintainability
\item Performance
\item Portability
\item Reliability
\item Reusability
\item Robustness
\item Scalability
\item Understandability
\item Usability
\end{enumerate}
\\
The requirements have been compared using a pairwise process, the results of which are listed in Table \ref{table:nfrpairwise} below. The comparison took into account the available documentation of current off the shelf LBS solutions, listed in Table \ref{table:otsinputs}, Table \ref{table:otscomp}, and Table \ref{table:otsparm}.

\begin{table}[!h]
\begin{center}
\begin{tabular}{| c | c | c | c | c | c | c | c | c | c | c | c |}
\hline
\pbox{1.25cm}{\textbf{NFR/ NFR} }& \textbf{1} & \textbf{2} & \textbf{3} & \textbf{4} & \textbf{5} & \textbf{6} & \textbf{7} & \textbf{8} & \textbf{9} & \textbf{10} & \sum \\
\hline
\textbf{1} & - & 1 & 1 & 1 & 1 & 1 & 1 & 1 & 1 & 1 & 9 \\
\hline
\textbf{2} & 0 & - & 0 & 1 & 0 & 1 & 1 & 1 & 0 & 0 & 4 \\
\hline
\textbf{3} & 0 & 1 & - & 1 & 0 & 1 & 0 & 1 & 1 & 1 & 6 \\
\hline
\textbf{4} & 0 & 0 & 0 & - & 0 & 0 & 0 & 0 & 0 & 0 & 0 \\
\hline
\textbf{5} & 0 & 1 & 1 & 1 & - & 1 & 1 & 1 & 1 & 1 & 8 \\
\hline
\textbf{6} & 0 & 0 & 0 & 1 & 0 & - & 0 & 0 & 0 & 0 & 1 \\
\hline
\textbf{7} & 0 & 0 & 1 & 1 & 0 & 1 & - & 1 & 1 & 0 & 5 \\
\hline
\textbf{8} & 0 & 0 & 0 & 1 & 0 & 1 & 0 & - & 0 & 0 & 2 \\
\hline
\textbf{9} & 0 & 1 & 0 & 1 & 0 & 1 & 0 & 1 & - & 0 & 4 \\
\hline
\textbf{10} & 0 & 1 & 0 & 1 & 0 & 1 & 1 & 1 & 1 & - & 6 \\
\hline
\end{tabular}
\caption{Pairwise Comparison of NFR}
\label{table:nfrpairwise}
\end{center}
\end{table}   

~\newpage

The following is a list of NFR by importance as found in the Table \ref{table:nfrpairwise} pairwise comparison:

\begin{enumerate}
\item Correctness
\item Reliability
\item Performance and Usability
\item Robustness
\item Maintainability and Understandability
\item Scalability
\item Reusability
\item Portability
\end{enumerate}

\section{Likely Changes}    

\noindent \begin{itemize}

\item[LC\refstepcounter{lcnum}\thelcnum\label{LC_output}:] 
A family of LBS solvers will have 2D and 3D output. 1D output is not a common variability. See Table \ref{table:otsinputs}.
\item[LC\refstepcounter{lcnum}\thelcnum\label{LC_wallpressure}:] 
Wall pressure is not an output variability that is often needed. This may be removed from a family of LBS. See Table \ref{table:otsparm}.
\item[LC\refstepcounter{lcnum}\thelcnum\label{LC_decomposiiton}:] 
Spinoidal decomposition is most common among LBS family members and should be the standard for a library implementation. See Table \ref{table:otscomp}.
\item[LC\refstepcounter{lcnum}\thelcnum\label{LC_parallel}:] 
MPI is the standard parallel interface for LBS and should be the standard for a library implementation. See Table \ref{table:otscomp}.
\item[LC\refstepcounter{lcnum}\thelcnum\label{LC_input}:] 
LBS generally read input parameters from a file and this should be the standard for a library implementation. See Table \ref{table:otsinputs}.

\end{itemize}

~\newpage

\section{Traceability Matrices and Graphs}

\begin{table}[!h]
\begin{center}
\begin{tabular}{| c | c | c | c | c | c | c | c | c |}
\hline
& \textbf{\tref{T_BTE} }& \textbf{\tref{T_BGK} } & \textbf{\tref{T_EDF} } & \textbf{DD\ref{DD_Velocity} } & \textbf{DD\ref{DD_Viscosity} } & \textbf{DD\ref{DD_RelaxationRate} } & \textbf{DD\ref{DD_VelocityGradient} } & \textbf{DD\ref{DD_FluidDensity} }\\
\hline
\textbf{\tref{T_BTE} }& & & & & & & &\\
\hline
\textbf{\tref{T_BGK} }& & & & & & & &\\
\hline
\textbf{\tref{T_EDF} }& & & & & & & &\\
\hline
\textbf{DD\ref{DD_Velocity} } & & & & & & & &\\
\hline
\textbf{DD\ref{DD_Viscosity} } & & & & & & & & \\
\hline
\textbf{DD\ref{DD_RelaxationRate} } & & & & & & & & \\
\hline
\textbf{DD\ref{DD_VelocityGradient} } & & & & & & & &\\
\hline
\textbf{DD\ref{DD_FluidDensity} } & & & & & & & &\\
\hline
\end{tabular}
\caption{Traceability Matrix Showing the Connections Between Items of Different Sections}
\end{center}
\end{table}   

\begin{table}[!h]
\begin{center}
\begin{tabular}{| c | c | c | c | c | c | c | c |}
\hline
& \textbf{A\ref{A_fluids} }& \textbf{A\ref{A_flowObject} } & \textbf{A\ref{A_lattice} } &  \textbf{A\ref{A_weightCoefficients} } & \textbf{A\ref{A_selectModel} } & \textbf{A\ref{A_userInputs} } & \textbf{DD\ref{A_speedSound} }\\
\hline
\textbf{\tref{T_BTE} }& & & & & & &\\
\hline
\textbf{\tref{T_BGK} }& & & & & & &\\
\hline
\textbf{\tref{T_EDF} }& & & & & & &\\
\hline
\textbf{DD\ref{DD_Velocity} } & & & & & & &\\
\hline
\textbf{DD\ref{DD_Viscosity} } & & & & & & &\\
\hline
\textbf{DD\ref{DD_RelaxationRate} } & & & & & & &\\
\hline
\textbf{DD\ref{DD_VelocityGradient} } & & & & & & &\\
\hline
\textbf{DD\ref{DD_FluidDensity} } & & & & & & &\\
\hline
\textbf{DD\ref{LC_output} } & & & & & & &\\
\hline
\textbf{DD\ref{LC_wallpressure} } & & & & & & &\\
\hline
\textbf{DD\ref{LC_decomposiiton} } & & & & & & &\\
\hline
\textbf{DD\ref{LC_parallel} } & & & & & & &\\
\hline
\textbf{DD\ref{LC_input} } & & & & & & &\\
\hline
\end{tabular}
\caption{Traceability Matrix Showing the Connections Between Assumptions and Other Items}
\end{center}
\end{table}   

\newpage

\bibliographystyle {plainnat}
\bibliography {../../refs/References}

\newpage

\section{Appendix}

\subsection{Symbolic Parameters}
\label{symbolicpara}

\begin{itemize}

\item[\label{Cons_SpeedSound}]Cons\_SpeedSound: The speed of sound $c_s$ is equal to 343 $\frac{m}{s}$. The binding time for this constant is during compile time. This constant is referenced by T\ref{T_EDF}, and is invoked by A\ref{A_speedSound}.

\end{itemize}

\subsection{Off The Shelf Solutions}
\label{OTSsolutions}

The following tables list some off the shelf Lattice Boltzmann Solvers, along with input parameters, computational parameters, and output parameters.

\begin{table}[!h]
\begin{center}
\begin{tabular}{| c | c | c | c | c | c | c | c |}
\hline
\textbf{solver} & \textbf{velocity} & \textbf{density} & \textbf{model} & \makecell{\textbf{velocity} \\ \textbf{directions}} & \textbf{time} & \textbf{viscosity} & \makecell{\textbf{input} \\ \textbf{method}}\\
\hline
hemeLB\cite{mazzeo2008hemelb} & $\geq$0 & $\geq$0 & 3D & 15 & $\geq$0 & $\geq$0 & prompt\\
\hline
MUPHY\cite{muphy} & $\geq$0 & $\geq$0 & 3D & 19 & $\geq$0 & $\geq$0 & file\\
\hline
Walberla\cite{schornbaum2016massivelyWaLBerla} & $\geq$0 & $\geq$0 & 2D/3D & 19 & $\geq$0 & $\geq$0 & file\\
\hline
DL\textunderscore Meso\cite{seaton2016dl} & $\geq$0 & $\geq$0 & 2D/3D & 9,15,19,27 & $\geq$0 & $\geq$0 & file\\
\hline
LB3D\cite{schmieschek2017lb3d} & $\geq$0 & $\geq$0 & 3D & 19 & $\geq$0 & $\geq$0 & file\\
\hline
Sailfish\cite{januszewski2014sailfish} & $\geq$0 & $\geq$0 & 2D/3D & \makecell{9,13,15, \\ 19,27} & $\geq$0 & &\\
\hline
mplabs\cite{mplabs} & $\geq$0 & $\geq$0 & 2D/3D & 9,19 & $\geq$0 & & file\\
\hline
LBSIM\cite{lbsim} & $\geq$0 &  & 2D/3D & 6,19 & $\geq$0 & &\\
\hline
pylbm\cite{pylbm} & $\geq$0 & $\geq$0 & 1D,2D,3D & \makecell{2,3,5,9,\\ 13,15,19} & $\geq$0 & $\geq$0 & file\\
\hline
\end{tabular}
\caption{OTS LBS Inputs}
\label{table:otsinputs}
\end{center}
\end{table}

\begin{table}[!h]
\begin{center}
\begin{tabular}{| c | c | c | c |}
\hline
\textbf{solver} & \textbf{computational model} & \textbf{decomposition technique} & \textbf{parallel interface} \\
\hline
hemeLB\cite{mazzeo2008hemelb} & D3Q15i & ParMETIS library & MPI \\
\hline
MUPHY\cite{muphy} & D3Q19+ & PT\textunderscore Scotch library & MPI \\
\hline
Walberla\cite{schornbaum2016massivelyWaLBerla} & D2Q9, D3Q19 & block-wide decomposition & MPI\\
\hline
DL\textunderscore Meso\cite{seaton2016dl} & \pbox{3cm}{D2Q9, D3Q15, D3Q19, D3Q27} & domain decomposition & MPI \\
\hline
LB3D\cite{schmieschek2017lb3d} & D3Q19 & spinodal decomposition & MPI \\
\hline
Sailfish\cite{januszewski2014sailfish} & \pbox{3cm}{D2Q9, D3Q13, D3Q15, D3Q19, D3Q27} & spinoidal decomposition & MPI\\
\hline
mplabs\cite{mplabs} & D2Q9, D3Q19 & & MPI \\
\hline
LBSIM\cite{lbsim} & D2Q6, D3Q19 & spinoidal decomposition & \\
\hline
pylbm\cite{pylbm} & \pbox{3cm}{ \\ D1Q2, D1Q3, D1Q5, D2Q9, D2Q13, D2Q15, D3Q15, D3Q19} & & MPI \\
\hline
\end{tabular}
\caption{OTS LBS Computational Parameters}
\label{table:otscomp}
\end{center}
\end{table}

\begin{table}[!h]
\begin{center}
\begin{tabular}{| c | c | c | c |}
\hline
\textbf{solver} & \textbf{wall pressure} & \textbf{flow velocity} & \textbf{graphical model} \\
\hline
hemeLB\cite{mazzeo2008hemelb} & $\geq$0 & $\geq$0 & 2D/3D \\
\hline
MUPHY\cite{muphy} & & & 2D/3D \\
\hline
Walberla\cite{schornbaum2016massivelyWaLBerla} & & $\geq$0 & 2D/3D\\
\hline
DL\textunderscore Meso\cite{seaton2016dl} & & $\geq$0 & 2D/3D \\
\hline
LB3D\cite{schmieschek2017lb3d} & & $\geq$0 & 2D/3D \\
\hline
Sailfish\cite{januszewski2014sailfish} & & $\geq$0 & 2D \\
\hline
mplabs\cite{mplabs} & & $\geq$0 & 2D/3D \\
\hline
LBSIM\cite{lbsim} & & & 2D/3D\\
\hline
pylbm\cite{pylbm} & & & 2D/3D \\
\hline
\end{tabular}
\caption{OTS LBS Output Parameters}
\label{table:otsparm}
\end{center}
\end{table}

~\newpage
~\newpage
\subsection{Coefficient Weights for Equilibrium Distribution Function}

\begin{table}[!h]
\begin{center}
\begin{tabular}{| c | c |}
\hline
\textbf{lattice model} & \textbf{coefficient weights} (w_i)\\
\hline
D1Q2\cite{} & \\
\hline
D1Q3\cite{} & \pbox{8cm}{4/6, i = 0; 1/6, i=1,2}\\
\hline
D1Q5\cite{} & \\
\hline
D2Q9\cite{perumal2015review} & \pbox{8cm}{4/9,i = 0; 1/9, i = 1,2,3,4; 1/36, i = 5,6,7,8}\\
\hline
D2Q13\cite{} & \\
\hline
D2Q15\cite{} & \\
\hline
D3Q15\cite{perumal2015review} & \pbox{10cm}{2/9, i = 0; 1/9, i = 1,2,...,6; 1/72, i = 7,8,...,14}\\
\hline
D3Q19\cite{perumal2015review} & \pbox{10cm}{2/9, i = 0; 1/18, i = 1,2,...,6; 1/36, i = 7,8,...,18}\\
\hline
D3Q27\cite{perumal2015review} & \pbox{6cm}{8/27, i = 0; 2/27, i = 1,2,...,6; 1/54, i = 7,8,...,18; 1/216. i = 19,20,...,26}\\
\hline


\end{tabular}
\caption{Lattice Model Coefficient Weights}
\label{coefficientweights}
\end{center}
\end{table}

\end{document}
