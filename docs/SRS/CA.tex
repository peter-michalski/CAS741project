\documentclass[12pt]{article}

\usepackage{amsmath, mathtools}
\usepackage{amsfonts}
\usepackage{amssymb}
\usepackage{graphicx}
\usepackage{colortbl}
\usepackage{xr}
\usepackage{hyperref}
\usepackage{longtable}
\usepackage{xfrac}
\usepackage{tabularx}
\usepackage{float}
\usepackage{siunitx}
\usepackage{booktabs}
\usepackage{caption}
\usepackage{pdflscape}
\usepackage{afterpage}
\usepackage{caption}
\usepackage{pbox}
\usepackage{makecell}




\usepackage[sort&compress,square,comma,numbers]{natbib}

\DeclareRobustCommand{\citeext}[1]{\citeauthor{#1}~\cite{#1}}

%\usepackage{refcheck}

\hypersetup{
    bookmarks=true,         % show bookmarks bar?
      colorlinks=true,       % false: boxed links; true: colored links
    linkcolor=red,          % color of internal links (change box color with linkbordercolor)
    citecolor=blue,        % color of links to bibliography
    filecolor=magenta,      % color of file links
    urlcolor=cyan           % color of external links
}

%% Comments

\usepackage{color}

\newif\ifcomments\commentstrue

\ifcomments
\newcommand{\authornote}[3]{\textcolor{#1}{[#3 ---#2]}}
\newcommand{\todo}[1]{\textcolor{red}{[TODO: #1]}}
\else
\newcommand{\authornote}[3]{}
\newcommand{\todo}[1]{}
\fi

\newcommand{\wss}[1]{\authornote{blue}{SS}{#1}} 
\newcommand{\plt}[1]{\authornote{magenta}{TPLT}{#1}} %For explanation of the template
\newcommand{\an}[1]{\authornote{cyan}{Author}{#1}}


% For easy change of table widths
\newcommand{\colZwidth}{1.0\textwidth}
\newcommand{\colAwidth}{0.13\textwidth}
\newcommand{\colBwidth}{0.82\textwidth}
\newcommand{\colCwidth}{0.1\textwidth}
\newcommand{\colDwidth}{0.05\textwidth}
\newcommand{\colEwidth}{0.8\textwidth}
\newcommand{\colFwidth}{0.17\textwidth}
\newcommand{\colGwidth}{0.5\textwidth}
\newcommand{\colHwidth}{0.28\textwidth}

% Used so that cross-references have a meaningful prefix
\newcounter{defnum} %Definition Number
\newcommand{\dthedefnum}{GD\thedefnum}
\newcommand{\dref}[1]{GD\ref{#1}}
\newcounter{datadefnum} %Datadefinition Number
\newcommand{\ddthedatadefnum}{DD\thedatadefnum}
\newcommand{\ddref}[1]{DD\ref{#1}}
\newcounter{theorynum} %Theory Number
\newcommand{\tthetheorynum}{T\thetheorynum}
\newcommand{\tref}[1]{T\ref{#1}}
\newcounter{tablenum} %Table Number
\newcommand{\tbthetablenum}{T\thetablenum}
\newcommand{\tbref}[1]{TB\ref{#1}}
\newcounter{assumpnum} %Assumption Number
\newcommand{\atheassumpnum}{P\theassumpnum}
\newcommand{\aref}[1]{A\ref{#1}}
\newcounter{goalnum} %Goal Number
\newcommand{\gthegoalnum}{P\thegoalnum}
\newcommand{\gsref}[1]{GS\ref{#1}}
\newcounter{instnum} %Instance Number
\newcommand{\itheinstnum}{IM\theinstnum}
\newcommand{\iref}[1]{IM\ref{#1}}
\newcounter{reqnum} %Requirement Number
\newcommand{\rthereqnum}{P\thereqnum}
\newcommand{\rref}[1]{R\ref{#1}}
\newcounter{lcnum} %Likely change number
\newcommand{\lthelcnum}{LC\thelcnum}
\newcommand{\lcref}[1]{LC\ref{#1}}

\newcommand{\famname}{Lattice Boltzmann Solvers} % PUT YOUR PROGRAM NAME HERE

\usepackage{fullpage}

\begin{document}

\title{\famname} 
\author{Peter Michalski}
\date{\today}

\maketitle

~\newpage

\pagenumbering{roman}

\section{Revision History}

\begin{tabularx}{\textwidth}{p{3cm}p{2cm}X}
\toprule {\bf Date} & {\bf Version} & {\bf Notes}\\
\midrule
October 7,2019 & 1.0 & Initial Document\\
\bottomrule
\end{tabularx}

~\newpage
	
\section{Reference Material}

This section records information for easy reference.

\subsection{Table of Units}

Throughout this document SI (Syst\`{e}me International d'Unit\'{e}s) is employed
as the unit system.  In addition to the basic units, several derived units are
used as described below.  For each unit, the symbol is given followed by a
description of the unit and the SI name.
~\newline

\renewcommand{\arraystretch}{1.2}
%\begin{table}[ht]
  \noindent \begin{tabular}{l l l} 
    \toprule		
    \textbf{symbol} & \textbf{unit} & \textbf{SI}\\
    \midrule 
    \si{\metre} & length & metre\\
    \si{\kilogram} & mass	& kilogram\\
    $t$ & time & second\\
    $F$ & force & newton\\
    $cm$ & length & centimetre\\
    $g$ & mass & gram \\
    \bottomrule
  \end{tabular}
  %	\caption{Provide a caption}
%\end{table}

\subsection{Table of Symbols}

The table that follows summarizes the symbols used in this document along with
their units.  The choice of symbols was made to be consistent with the heat
transfer literature and with existing documentation for solar water heating
systems.  The symbols are listed in alphabetical order.

\renewcommand{\arraystretch}{1.2}
%\noindent \begin{tabularx}{1.0\textwidth}{l l X}
\noindent \begin{longtable*}{l l p{12cm}} \toprule
\textbf{symbol} & \textbf{unit} & \textbf{description}\\
\midrule 
$e$ & $\frac{m}{s}$ & velocity
\\
$\eta$ & $Pa-s$ & viscosity
\\ 
$A$ & $m^2$ & cross-sectional area
\\
$\gamma$ & $\frac{1}{s}$ & velocity gradient
\\
$\tau$ & N/A & relaxation rate
\\
$x$ & N/A & position vector
\\
$f$ & N/A & distribution function
\\
$\Omega$ & N/A & collision operator
\\
$f^{eq}$ & N/A & equilibrium distribution function
\\
$k$ & N/A & velocity direction
\\
$p$ & $\frac{g}{cm^3}$ & fluid density
\\
$w$ & N/A & weight coefficient (implementation specific)
\\
$u$ & $\frac{m}{s}$ & macroscopic velocity of fluid
\\
$\mathrm{D}$ & N/A & signifies the dimension component of lattice model
\\
$\mathrm{Q}$ & N/A & signifies number of velocity directions of lattice model
\\
$\sigma$ & N/A & variable number of dimensions in the lattice model
\\
$\kappa$ & N/A & variable number of velocity directions of lattice model, also referred to as linkages
\\
\bottomrule
\end{longtable*}

\subsection{Abbreviations and Acronyms}

\renewcommand{\arraystretch}{1.2}
\begin{tabular}{l l} 
  \toprule		
  \textbf{symbol} & \textbf{description}\\
  \midrule
  1D & 1-Dimensional\\ 
  2D & 2-Dimensional\\ 
  3D & 3-Dimensional\\ 
  A & Assumption\\
  CA & Commonality Analysis\\
  DD & Data Definition\\
  GS & Goal Statement\\
  LBM & Lattice Boltzmann Methods\\
  LBS & Lattice Boltzmann Solvers\\
  LC & Likely Change\\
  PS & Physical System Description\\
  R & Requirement\\
  T & Theoretical Model\\
  \bottomrule
\end{tabular}\\

\newpage

\tableofcontents

~\newpage

\pagenumbering{arabic}

\section{Introduction}

This document provides a Commonality Analysis (CA) for a family of Lattice Boltzmann Solvers (LBS), which provide services based on Lattice Boltzmann Methods (LBM).
LBM are a family of fluid dynamics algorithms for simulating single-phase and multiphase fluid flows, often incorporating additional physical complexities. \citet{chen1998lattice}. They consider the behaviours of a collection of particles as a single "unit" at the mesoscopic scale. These methods predict the positional probability of a collection of particles moving through a lattice structure. Various off the shelf Lattice Boltzmann Solvers (LBS) solutions available today allow for a range of fluid and physical model input parameters, computational parameters, and output parameters as outlined in Section \ref{OTSsolutions}.
The following subsection of this introduction will outline the purpose of this document, a general scope of the family of LBS, the characteristics of the intended reader, and finally an outline of the rest of this document.

\subsection{Purpose of Document}

The purpose of this document is to provide general information on the currently available LBS solutions, including their commonalities and variabilities, as well as a baseline understanding of the model and structure of abstract LBM. The information provided here will be used in the development of the design of a solution providing services of a family of LBS.

\subsection{Scope of the Family} 

The family of LBS will model one or more fluids as they pass through a boundary, modeled my a lattice. Fluids with any properties can be modeled, however only those properties that are accepted as inputs by the LBS will affect the model results. The calculation of the LBM distribution function will use up to 3D computational models, and will output the data into memory and render it in up to 3D imaging.

\subsection{Characteristics of Intended Reader} 

The intended reader of this document should have an undergraduate understanding of software requirements and specifications as well as software design principles. Ideally, the user will be knowledgeable of commonality analysis, 

\subsection{Organization of Document}

This document is organized along a template for a CA for scientific computing software proposed by \citet{smith2006systematic}. It follows a standard pattern of presenting a general system description, commonalities of the members of the software family, variabilities of the members of the software family, and the requirements for the family of LBS. The goal statements of the family of LBS, found in section \ref{goalstatements},are refined to the theoretical models in Section \ref{sec_theoretical}. Variabilities within the family are found in \ref{variabilities}. Tables of off the shelf solution commonalities and variabilities are found in Section \ref{OTSsolutions}.

\section{General System Description}

This section identifies the interfaces between the system and its environment,
describes the potential user characteristics and lists the potential system
constraints.

\subsection{Potential System Contexts}

\begin{itemize}
\item User Responsibilities:
\begin{itemize}
\item The user must provide the system with correctly formatted physical model parameters.
\item The user must select the desired mathematical model for the computation.
\item The user must select the desired format of output for the model.
\end{itemize}
\item \famname{} Responsibilities:
\begin{itemize}
\item Detect data type mismatch, such as a negative number instead of a positive number for a parameter, such as $A$ that cannot accept negative values.
\item Initialize the correct data types and data structures for the model.
\item Perform the calculations to predict the distribution of fluid particles over time.
\item Store the distribution function output data.
\item Store calculated fluid parameters over time.
\item Visually model the results of the distribution function.
\item Store the calculation results in a file and/or in memory.
\item Detect errors during parameter input, model calculation, or model output; store the errors in a file and show the error to the user.
\item Recover from error states, such as those that develop from division by zero or a buffer overflow.
\end{itemize}
\end{itemize}

\subsection{Potential User Characteristics} \label{SecUserCharacteristics}

The end user of \famname{} should ideally have an understanding of undergraduate Level 1 Physics and Fluid Dynamics. The ideal end user characteristics may variate between the specific members of the family of solvers. For example, a user of HemeLB, a off the shelf LBM solution for simulating blood flow, would ideally have an understanding of phlebology.

\subsection{Potential System Constraints}

The parallel nature of LBS prefers operating and hardware systems that can handle concurrency and large amounts of data. Modern operating systems and computer hardware platforms are suggested. Memory should be scaled to the requirements of the desired LBS library.

\section{Commonalities}

\subsection{Background Overview} \label{Sec_Background}

As LBS model fluid dynamics within a boundary using a predefined lattice structure, the methods rely on a two step calculation process. The first processes is known as streaming, where the particles move along the lattice via links, and the second process is collision, where energy and momentum is transferred among particles that collide \cite{bao2011lattice}.
In the LBS solutions, the particles are mapped using a lattice structure. The lattice structure can be a 1D, 2D, or 3D model with varying velocity directions. The notation is D$\sigma$Q$\kappa$, where $\sigma$ represents the number of dimensions and $\kappa$ represents the number of velocity directions.There are many standardized lattice models; individual solvers within the family may only use a subset of them.
The LBM uses the initial parameters of the fluid to find the probability of where along the lattice linkages a group of particles are most likely to travel. It then moves the particles into the next node, and transfers the energy and momentum if a collision occurs. Then the process repeats for the duration of the modeling instance.

\subsection{Terminology and  Definitions}

This subsection provides a list of terms that are used in the subsequent
sections and their meaning, with the purpose of reducing ambiguity and making it
easier to correctly understand the requirements:


also see slide 35?

\begin{itemize}

\item 

\end{itemize}

\subsection{Data Definitions} \label{sec_datadef}

This section collects and defines all the data needed to build the instance models. The dimension of each quantity is also given.  

~\newline

\noindent
\begin{minipage}{\textwidth}
\renewcommand*{\arraystretch}{1.5}
\begin{tabular}{| p{\colAwidth} | p{\colBwidth}|}
\hline
\rowcolor[gray]{0.9}
Number& DD\refstepcounter{datadefnum}\thedatadefnum \label{DD_Velocity}\\
\hline
Label& \bf Velocity\\
\hline
Symbol &$\mathrm{e}$\\
\hline
% \hline
  SI Units & \si{\frac\metre\second}\\
  \hline
  Equation& $\mathrm{e} = \frac{d \mathrm{r}}{dt}$\\
  \hline
  Description & 
                 Velocity is the distance that an object moves relative to time. $r$ is the the distance ($\mathrm{m}$) in change for our change in time {t} of units ($\mathrm{s}$).
  \\
  \hline
  Sources& \citet{mohamad2011lattice}\\
  \hline
  Ref.\ By & \tref{T_BTE} \tref{T_EDF}\\
  \hline
\end{tabular}
\end{minipage}\\


~\newline


\noindent
\begin{minipage}{\textwidth}
\renewcommand*{\arraystretch}{1.5}
\begin{tabular}{| p{\colAwidth} | p{\colBwidth}|}
\hline
\rowcolor[gray]{0.9}
Number& DD\refstepcounter{datadefnum}\thedatadefnum \label{DD_Viscosity}\\
\hline
Label& \bf Viscosity\\
\hline
Symbol &$\mathrm{\eta}$\\
\hline
% \hline
  SI Units & $\mathrm{Pa-s}$\\
  \hline
  Equation& $\mathrm{\eta} = \frac{F/A}{\gamma}$\\
  \hline
  Description & 
                Viscosity is the measure of resistance to deformation. $F$ is the applied force (N), $A$ is the cross-sectional area ($m^2$), and $\gamma$ is the velocity gradient. 
  \\
  \hline
  Sources& \citet{viscosity}\\
  \hline
  Ref.\ By & DD\ref{DD_RelaxationRate}\\
  \hline
\end{tabular}
\end{minipage}\\

~\newline

\noindent
\begin{minipage}{\textwidth}
\renewcommand*{\arraystretch}{1.5}
\begin{tabular}{| p{\colAwidth} | p{\colBwidth}|}
\hline
\rowcolor[gray]{0.9}
Number& DD\refstepcounter{datadefnum}\thedatadefnum \label{DD_RelaxationRate}\\
\hline
Label& \bf Relaxation Rate Towards Equilibrium\\
\hline
Symbol &$\tau$\\
\hline
% \hline
  SI Units & NA\\
  \hline
  Equation&$\tau = \frac{12\mathrm{\eta}\Delta t}{\Delta\mathrm{x}^2} + \frac{1}{2}$\\
  \hline
  Description & 
                The relaxation rate defines how quickly the particles recover to equilibrium state. Adjusting this method in the implementation allows for the simulation of complex physical phenomena, specifically concerning the fluid media. $\mathrm{\eta}$ is the viscosity of the fluid, $t$ is the time interval (s), and $x$ is the position vector.
  \\
  \hline
  Sources& \citet{lbmbolton}\\
  \hline
  Ref.\ By & \tref{T_BTE}\\
  \hline
\end{tabular}
\end{minipage}\\

~\newline

\noindent
\begin{minipage}{\textwidth}
\renewcommand*{\arraystretch}{1.5}
\begin{tabular}{| p{\colAwidth} | p{\colBwidth}|}
\hline
\rowcolor[gray]{0.9}
Number& DD\refstepcounter{datadefnum}\thedatadefnum \label{DD_VelocityGradient}\\
\hline
Label& \bf Velocity Gradient\\
\hline
Symbol &$\gamma$\\
\hline
% \hline
  SI Units &$\frac{1}{s}$\\
  \hline
  Equation&$\gamma = \frac{d\mathrm{e}}{dz}$\\
  \hline
  Description & 
                Velocity gradient is the difference in velocity between adjacent fluids. $d\mathrm{e}$ represents the difference in velocities of the fluids and $dz$ is the
distance of the two velocities.  \\
  \hline
  Sources& \citet{viscosity}\\
  \hline
  Ref.\ By & DD\ref{DD_Viscosity}\\
  \hline
\end{tabular}
\end{minipage}\\

~\newline

\noindent
\begin{minipage}{\textwidth}
\renewcommand*{\arraystretch}{1.5}
\begin{tabular}{| p{\colAwidth} | p{\colBwidth}|}
\hline
\rowcolor[gray]{0.9}
Number& DD\refstepcounter{datadefnum}\thedatadefnum \label{DD_FluidDensity}\\
\hline
Label& \bf Fluid Density\\
\hline
Symbol &$p$\\
\hline
% \hline
  SI Units &$\frac{g}{cm^3}$ \\
  \hline
  Equation& $p$ = $\frac{g}{cm^3}$ \\
  \hline
  Description & 
                Density is the ratio of mass to volume of a material. $g$ is the mass and $cm^3$ is the volume.  \\
  \hline
  Sources& \citet{density}\\
  \hline
  Ref.\ By & \tref{T_EDF}\\
  \hline
\end{tabular}
\end{minipage}\\


\subsection{Goal Statements}
\label{goalstatements}
\noindent Given the boundary conditions, lattice model, weighting coefficient of the lattice, simulation time, fluid particle mass, and initial conditions for the momentum, density and position of the fluid particles, as well as any applied external force, the goal statements are:

\begin{itemize}

\item[\label{G_Probability}]G\_prob: Predict the location probabilities of fluid particles in the lattice over time.

\item[\label{G_Model}]G\_model: Model the location of fluid particles within the lattice over time.

\item[\label{G_Velocity}]G\_velocity: Model the velocity of fluid particles within the lattice over time.

\item[\label{G_FluidPressure}]G\_fluidPressure: Model the pressure of fluid particles within the lattice over time.

\item[\label{G_WallPressure}]G\_wallPressure: Model the pressure exerted on the walls of the boundary over time.

\end{itemize}

\subsection{Theoretical Models} \label{sec_theoretical}

This section focuses on the general equations and laws that \famname{} are based
on.  

~\newline

\noindent
\begin{minipage}{\textwidth}
\renewcommand*{\arraystretch}{1.5}
\begin{tabular}{| p{\colAwidth} | p{\colBwidth}|}
  \hline
  \rowcolor[gray]{0.9}
  Number& T\refstepcounter{theorynum}\thetheorynum \label{T_BTE}\\
  \hline
  Label&\bf Boltzmann Transport Equation\\
  \hline
  Equation&  $f(\mathrm{x} +\mathrm{e}dt, \mathrm{e} + \frac{\mathrm{F}}{\mathrm{m}}dt, t + dt)d\mathrm{x}d\mathrm{e} - f(\mathrm{x},\mathrm{e},t)d\mathrm{x}d\mathrm{e} = \mathrm{\Omega}(f)d\mathrm{x}d\mathrm{e}$\\
  \hline
  Description & 
  This equation determines the statistical description of a group of particles. The left part of the equation, $f(\mathrm{x} +\mathrm{e}dt, \mathrm{e} + \frac{\mathrm{F}}{\mathrm{kg}}dt, t + dt)d\mathrm{x}d\mathrm{e}$, represents the distribution function result after an external force $\mathrm{F}$ is applied. The middle function, $f(\mathrm{x},\mathrm{e},t)d\mathrm{x}d\mathrm{e}  $, represents the distribution function result before the external force is applied. The distribution function $f$ represents the probability that a set of particles will be at a specific location of the lattice at a specified time. The right hand side of the equation represents the collision operator, $\Omega$.\newline
The variable $\mathrm{x}$ represents the vector of the particles within the lattice, $\mathrm{e}$ is velocity $\mathrm{\frac{m}{s}}$, $\mathrm{t}$ is time ($\mathrm{s}$), $\mathrm{F}$ is force ($\mathrm{N}$), $\mathrm{kg}$ is length ($\mathrm{kg}$). This equation can be further developed for specific instances.\\
  \hline
  Source &
           \url{https://personal.ems.psu.edu/~fkd/courses/EGEE520/2017Deliverables/LBM_2017.pdf}\newline \citet{mohamad2011lattice}\\
  % The above web link should be replaced with a proper citation to a publication
  \hline
  Ref.\ By & \\
  \hline
\end{tabular}
\end{minipage}\\

~\newline

\noindent
\begin{minipage}{\textwidth}
\renewcommand*{\arraystretch}{1.5}
\begin{tabular}{| p{\colAwidth} | p{\colBwidth}|}
  \hline
  \rowcolor[gray]{0.9}
  Number& T\refstepcounter{theorynum}\thetheorynum \label{T_BGK}\\
  \hline
  Label&\bf Bhatnagar-Gross-Krook Collision Operator\\
  \hline
  Equation&  $\mathrm{\Omega} = \frac{\Delta t}{\tau}(f^{eq}(r,t)-f(r,t))$\\
  \hline
  Description &
  The above equation is a mathematical operator that preserves continuity for a discretized model.
  $\tau$ is the relaxation rate towards equilibrium and should be in the range of 0.5 - 2.0. It is related to viscosity as outlined in DD. 
  $f^{eq}$ is the equilibrium particle probability distribution function. $f$ is the particle probability distribution function. This equation can be further developed for specific instances.\\
  \hline
  Source &
           \url{https://personal.ems.psu.edu/~fkd/courses/EGEE520/2017Deliverables/LBM_2017.pdf}\newline \citet{mohamad2011lattice}\\
  % The above web link should be replaced with a proper citation to a publication
  \hline
  Ref.\ By & \tref{T_BTE}\\
  \hline
\end{tabular}
\end{minipage}\\

~\newline

\noindent
\begin{minipage}{\textwidth}
\renewcommand*{\arraystretch}{1.5}
\begin{tabular}{| p{\colAwidth} | p{\colBwidth}|}
  \hline
  \rowcolor[gray]{0.9}
  Number& T\refstepcounter{theorynum}\thetheorynum \label{T_EDF}\\
  \hline
  Label&\bf Equilibrium Distribution Function\\
  \hline
  Equation&  $f_{k}^{eq} = pw_{k}[1 + \frac{2\overrightarrow{c}\overrightarrow{u}-\overrightarrow{u}\overrightarrow{u}}{2c_s^2}+\frac{(\overrightarrow{c}\overrightarrow{u})^2}{2c_s^4}] + O(u^2)$\\
  \hline
  Description &
  The above equation captures the probability distribution of the particles. Adjusting this method in the implementation allows for the simulation of complex physical phenomena, including geometry of the boundary.
  $p$ is the fluid density $(\mathrm{\frac{g}{cm^2}}$). $w$ is the weighting coefficient for the lattice model. $k$ is the discretized velocity direction, referring to the directions of the chosen lattice model. $\mathrm{e}$ is the velocity ($\mathrm{\frac{m}{s}}$). $\mathrm{u}$ is the macroscopic velocity of the fluid, which is a vector field of velocity at a specific position and time. This equation can be further developed for specific instances.\\
  \hline
  Source &
           \url{https://personal.ems.psu.edu/~fkd/courses/EGEE520/2017Deliverables/LBM_2017.pdf}\newline \citep{mohamad2011lattice}
           \newline \cite{mohamad2011lattice}
           \newline \citet{mohamad2011lattice}
           \\
  % The above web link should be replaced with a proper citation to a publication
  \hline
  Ref.\ By & \tref{T_BGK}\\
  \hline
\end{tabular}
\end{minipage}\\

~\newline

\section{Variabilities}
\label{variabilities}
state the binding time for each of the variabilities

input variabilities? what are their associated parameters of variation?  see table lecture 5 slide 39/40

\begin{table}[!h]
\begin{center}
\begin{tabular}{| c | c |}
\hline
Variability & Parameter of Variation\\
\hline
"Border" Shape & Set of {defined 2D, defined 3D, undefined}\\
\hline
"Border" Parameters & Set of {deflective, non deflective}\\
\hline
Fluid Parameters & Set of {$e$, $t$, $u$, $p$, $x$, $\eta$, $\tau$, $\gamma$, $F$, $A$}\\
\hline
Model Choice & Set of {1D, 2D, 3D}\\
\hline
Velocity Directions & Set of {2, 3, 5, 9,  13, 15, 19, 27}\\
\hline
Input Methods & Set of {file, type}\\
\hline
\end{tabular}
\caption{Input Variabilities}
\end{center}
\end{table}




\plt{The variabilities are summarized in the following subsections.  They may
  each be summarized separately, like in \citet{SmithMcCutchanAndCarette2017}, or
  in a table, as in \citet{Smith2006}.}

\plt{For each variability, a description should be given, along with the
  parameters of variation and the binding time.  The parameters of variation
  give the type that defines possible values.  The binding time is when the
  variability is set.  The possible values are specification time (scope time),
  build time and run time.}

\subsection{Assumptions}
also see slide 47 


\begin{itemize}

\item[A\refstepcounter{assumpnum}\theassumpnum \label{A_fluids}:]
  One or more fluids can be modeled.
  
\item[A\refstepcounter{assumpnum}\theassumpnum \label{A_flowObject}:]
  The fluid can, but does not need to, flow through an object with boundary conditions.
  
\item[A\refstepcounter{assumpnum}\theassumpnum \label{A_lattice}:]
  The fluid flows through space via a lattice structure, moving between lattice nodes via linkages $Q$.
  
\item[A\refstepcounter{assumpnum}\theassumpnum \label{A_visualPresentation}:]
  There is a visual presentation of the predicted fluid flow available in all libraries.
  
\item[A\refstepcounter{assumpnum}\theassumpnum \label{A_dataPassed}:]
  Data representing the predicted fluid flow can be passed externally of the system through memory or a file.  
  
\item[A\refstepcounter{assumpnum}\theassumpnum \label{A_weightCoefficients}:]
  Weight coefficients are standard for each lattice model. See the Appendix section for weight coefficients for a subset of lattice models.  

 table with goals and theoretical models on y axis and assumptions on x axis?
\item[A\refstepcounter{assumpnum}\theassumpnum \label{A_meaningfulLabel}:]
  \plt{Short description of each assumption.  Each assumption
    should have a meaningful label.  Use cross-references to identify the
    appropriate traceability to T, GD, DD etc., using commands like dref, ddref etc.}

 table with goals and theoretical models on y axis and assumptions on x axis

\end{itemize}

\plt{Input assumptions will be appropriate for many problems.  Some input will
  have simplifying constraints, and other inputs will not.}

\subsection{Calculation} \label{sec_Calculation}

\begin{table}[!h]
\begin{center}
\begin{tabular}{| c | c |}
\hline
Variability & Parameter of Variation\\
\hline
\end{tabular}
\caption{Calculation Variabilities}
\end{center}
\end{table}

\plt{The calculation variabilities should be as abstract as possible.  If there
  are variabilities that are related to imposed design decisions, the system
  constraints section should be referenced for the relevant constraint.  Design
  constraint related variabilities should be listed separately.}

\plt{Variabilities related to data structure choices would go in this section.
  However, these variabilities are related to design, so they should be
  separated from the more abstract variabilities.}

\plt{Algorithmic variations would go here as well, but as for data structures,
  they should be separated from the more abstract variabilities.}

\subsection{Output} \label{sec_Output} 

\begin{table}[!h]
\begin{center}
\begin{tabular}{| c | c |}
\hline
Variability & Parameter of Variation\\
\hline
\end{tabular}
\caption{Output Variabilities}
\end{center}
\end{table}   

\section{Requirements}

This section provides the functional requirements, the business tasks that the
software is expected to complete, and the nonfunctional requirements, the
qualities that the software is expected to exhibit.

\subsection{Family of Functional Requirements}

\plt{Since the CA will often be applied to a library, the functionality will not
  be a single use case.  Therefore, this section should summarize the family of
  potential requirements.  A good way to provide an overview of the functional
  requirements would be to provide multiple use cases on how the library will be
  employed.}

\noindent \begin{itemize}

\item[R\refstepcounter{reqnum}\thereqnum \label{R_Inputs}:] \plt{Requirements
    for the inputs that are supplied by the user.  This information has to be
    explicit.}

\item[R\refstepcounter{reqnum}\thereqnum \label{R_OutputInputs}:] \plt{It isn't
    always required, but often echoing the inputs as part of the output is a
    good idea.}

\item[R\refstepcounter{reqnum}\thereqnum \label{R_Calculate}:] \plt{Calculation
    related requirements.}

\item[R\refstepcounter{reqnum}\thereqnum \label{R_VerifyOutput}:]
  \plt{Verification related requirements.}

\item[R\refstepcounter{reqnum}\thereqnum \label{R_Output}:] \plt{Output related
    requirements.}

\end{itemize}

\subsection{Nonfunctional Requirements}

rather than absolute quantification of nfrs, use relative comparison between other program family members

specify requirements in big O notation

NFR -> AHP 

relative comparison between programs is a validateable requirement

focus on a posteriori description rather than a priori specification

identify benchmark test problems

test cases built starting from assumed solutions

see slide 42

ahp each design against each nfr see slide 42



\plt{To allow the Non-Functional Requirements (NFRs) to vary between family
  members, try to parameterize them.  The value of the parameter is than a variability.}

\plt{An important variability between family members it the relative importance
  of the NFRs.  \citet{Smith2006} shows how pairwise comparisons can be used to
  rank the importance of NFRs.}

\plt{List your nonfunctional requirements.  You may consider using a fit
  criterion to make them verifiable.}

Correctness - satisfies reqs spec
Reliability - usually does what it is intnded to do
Robustness - behaves reasnably during exceptional situations
Performance (low computer resource usage): evaluated using empirical measurement, analysis of an analytic model, or analysis of a simulation model
Usability - ease of usage, user interface,installability?
maintainability - ease of modification; corrective, adaptive, perfective
Reusability - create a new product? standardized? instantiable components?
Portability - run in different environments? hardware platform, operating system, supporting software, user base
add more
understandability - ease with which the reqs, design, implementation, documentation, etc can be understood -> impacts verifiability, maintainability, resusablity

Define these in the document before using them.



\section{Likely Changes}    

\noindent \begin{itemize}

\item[LC\refstepcounter{lcnum}\thelcnum\label{LC_meaningfulLabel}:] \plt{If
    there is a ranking of variabilities, or combinations of variabilities, that
    are more likely, this information can be included here.}

\end{itemize}

\section{Traceability Matrices and Graphs}

\plt{You will have to add tables.}



\newpage

\bibliographystyle {plainnat}
\bibliography {../../refs/References}

\newpage

\section{Appendix}

\plt{Your report may require an appendix.  For instance, this is a good point to
show the values of the symbolic parameters introduced in the report.}

\subsection{Symbolic Parameters}

\plt{The definition of the requirements will likely call for SYMBOLIC\_CONSTANTS.
Their values are defined in this section for easy maintenance.}

\noindent \plt{Advice on using the template:
\begin{itemize}
\item Assumptions have to be invoked somewhere
\item ``Referenced by'' implies that there is an explicit reference
\item Think of traceability matrix, list of assumption invocations and list of
  reference by fields as automatically generatable
\item If you say the format of the output (plot, table etc), then your
  requirement could be more abstract
\item For families the notion of binding time should be introduced
\item Think of families as a library, not as a single program
\end{itemize}
}

\subsection{Off The Shelf Solutions}
\label{OTSsolutions}
\subsubsection*{OTS1}
The following table lists some Lattice Boltzmann Solvers, along with an incomplete list of some input parameters. Cells that are blank represent an unknown value. Assumptions have been stated in the assumptions section.

\begin{table}[!h]
\begin{center}
\begin{tabular}{| c | c | c | c | c | c | c | c |}
\hline
solver & velocity & density & model & \makecell{velocity \\ directions} & time & viscosity & \makecell{input \\ method}\\
\hline
hemeLB\cite{mazzeo2008hemelb} & $\geq$0 & $\geq$0 & 3D & 15 & $\geq$0 & $\geq$0 & prompt\\
\hline
MUPHY\cite{muphy} & $\geq$0 & $\geq$0 & 3D & 19 & $\geq$0 & $\geq$0 & file\\
\hline
Walberla\cite{schornbaum2016massivelyWaLBerla} & $\geq$0 & $\geq$0 & 2D/3D & 19 & $\geq$0 & $\geq$0 & file\\
\hline
DL\textunderscore Meso\cite{seaton2016dl} & $\geq$0 & $\geq$0 & 2D/3D & 9,15,19,27 & $\geq$0 & $\geq$0 & file\\
\hline
LB3D\cite{schmieschek2017lb3d} & $\geq$0 & $\geq$0 & 3D & 19 & $\geq$0 & $\geq$0 & file\\
\hline
Sailfish\cite{januszewski2014sailfish} & $\geq$0 & $\geq$0 & 2D/3D & \makecell{9,13,15, \\ 19,27} & $\geq$0 & &\\
\hline
mplabs\cite{mplabs} & $\geq$0 & $\geq$0 & 2D/3D & 9,19 & $\geq$0 & & file\\
\hline
LBSIM\cite{lbsim} & $\geq$0 &  & 2D/3D & 6,19 & $\geq$0 & &\\
\hline
pylbm\cite{pylbm} & $\geq$0 & $\geq$0 & 1D,2D,3D & \makecell{2,3,5,9,\\ 13,15,19} & $\geq$0 & $\geq$0 & file\\
\hline
\end{tabular}
\caption{LBS Inputs}
\end{center}
\end{table}

\subsubsection{•}
The following table lists some Lattice Boltzmann Solvers, along with an incomplete list of some computational parameters. Cells that are blank represent an unknown value. Assumptions have been stated in the assumptions section.

\begin{table}[!h]
\begin{center}
\begin{tabular}{| c | c | c | c |}
\hline
solver & computational model & decomposition technique & parallel interface \\
\hline
hemeLB\cite{mazzeo2008hemelb} & D3Q15i & ParMETIS library & MPI \\
\hline
MUPHY\cite{muphy} & D3Q19+ & PT\textunderscore Scotch library & MPI \\
\hline
Walberla\cite{schornbaum2016massivelyWaLBerla} & D2Q9, D3Q19 & block-wide decomposition & MPI\\
\hline
DL\textunderscore Meso\cite{seaton2016dl} & \pbox{3cm}{D2Q9, D3Q15, D3Q19, D3Q27} & domain decomposition & MPI \\
\hline
LB3D\cite{schmieschek2017lb3d} & D3Q19 & spinodal decomposition & MPI \\
\hline
Sailfish\cite{januszewski2014sailfish} & \pbox{3cm}{D2Q9, D3Q13, D3Q15, D3Q19, D3Q27} & spinoidal decomposition & MPI\\
\hline
mplabs\cite{mplabs} & D2Q9, D3Q19 & & MPI \\
\hline
LBSIM\cite{lbsim} & D2Q6, D3Q19 & spinoidal decomposition & \\
\hline
pylbm_{\cite{pylbm}} & \pbox{3cm}{ \\ D1Q2, D1Q3, D1Q5, D2Q9, D2Q13, D2Q15, D3Q15,    D3Q19} & & MPI \\
\hline
\end{tabular}
\caption{LBS Computational Parameters}
\end{center}
\end{table}

The following table lists some Lattice Boltzmann Solvers, along with an incomplete list of some output parameters.

 Cells that are blank represent an unknown value. Assumptions have been stated in the assumptions section.

\begin{table}[!h]
\begin{center}
\begin{tabular}{| c | c | c | c |}
\hline
solver & wall pressure & flow velocity & graphical model \\
\hline
hemeLB\cite{mazzeo2008hemelb} & $\geq$0 & $\geq$0 & 2D/3D \\
\hline
MUPHY\cite{muphy} & & & 2D/3D \\
\hline
Walberla\cite{schornbaum2016massivelyWaLBerla} & & $\geq$0 & 2D/3D\\
\hline
DL\textunderscore Meso\cite{seaton2016dl} & & $\geq$0 & 2D/3D \\
\hline
LB3D\cite{schmieschek2017lb3d} & & $\geq$0 & 2D/3D \\
\hline
Sailfish\cite{januszewski2014sailfish} & & $\geq$0 & 2D \\
\hline
mplabs\cite{mplabs} & & $\geq$0 & 2D/3D \\
\hline
LBSIM\cite{lbsim} & & & 2D/3D\\
\hline
pylbm\cite{pylbm} & & & 2D/3D \\
\hline
\end{tabular}
\caption{LBS Output Parameters}
\end{center}
\end{table}

\subsection{Coefficient Weights for Equilibrium Distribution Function}

Blank coefficients are unknown to the author at the time of writing.

\begin{table}[!h]
\begin{center}
\begin{tabular}{| c | c |}
\hline
lattice model & coefficient weights (w_i)\\
\hline
D1Q2\cite{} & \\
\hline
D1Q3\cite{} & \pbox{8cm}{4/6, i = 0; 1/6, i=1,2}\\
\hline
D1Q5\cite{} & \\
\hline
D2Q9\cite{perumal2015review} & \pbox{8cm}{4/9,i = 0; 1/9, i = 1,2,3,4; 1/36, i = 5,6,7,8}\\
\hline
D2Q13\cite{} & \\
\hline
D2Q15\cite{} & \\
\hline
D3Q15\cite{perumal2015review} & \pbox{10cm}{2/9, i = 0; 1/9, i = 1,2,...,6; 1/72, i = 7,8,...,14}\\
\hline
D3Q19\cite{perumal2015review} & \pbox{10cm}{2/9, i = 0; 1/18, i = 1,2,...,6; 1/36, i = 7,8,...,18}\\
\hline
D3Q27\cite{perumal2015review} & \pbox{6cm}{8/27, i = 0; 2/27, i = 1,2,...,6; 1/54, i = 7,8,...,18; 1/216. i = 19,20,...,26}\\
\hline


\end{tabular}
\caption{Lattice Model Coefficient Weights}
\end{center}
\end{table}

\end{document}
