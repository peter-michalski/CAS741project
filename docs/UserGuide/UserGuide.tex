\documentclass[12pt, titlepage]{article}


\usepackage{amsmath, mathtools}
\usepackage{amsfonts}
\usepackage{amssymb}


\usepackage{caption}
\usepackage{pdflscape}
\usepackage{afterpage}
\usepackage{caption}
\usepackage{pbox}
\usepackage{makecell}


\usepackage{booktabs}
\usepackage{tabularx}
\usepackage{hyperref}
\hypersetup{
    colorlinks,
    citecolor=black,
    filecolor=black,
    linkcolor=red,
    urlcolor=blue
}
\usepackage{xr}
\externaldocument[ext1-]{../SRS/CA}
\externaldocument[ext2-]{../Design/MG/MG}
\externaldocument[ext3-]{../Design/MIS/MIS}
\externaldocument[ext4-]{../VnVPlan/SystVnVPlan/SystVnVPlan}
\externaldocument[ext4-]{../VnVPlan/UnitVnVPlan/UnitVnVPlan}
\usepackage[sort&compress,square,comma,numbers]{natbib}
\newcounter{reqnum} %Requirement Number
\newcommand{\myprogname}{Lattice Boltzmann Solver} 
\newcounter{uvtestcounter} %Test Number
\newcommand{\atheuvtestcounter}{P\theuvtestcounter}

%% Comments

\usepackage{color}

\newif\ifcomments\commentstrue

\ifcomments
\newcommand{\authornote}[3]{\textcolor{#1}{[#3 ---#2]}}
\newcommand{\todo}[1]{\textcolor{red}{[TODO: #1]}}
\else
\newcommand{\authornote}[3]{}
\newcommand{\todo}[1]{}
\fi

\newcommand{\wss}[1]{\authornote{blue}{SS}{#1}} 
\newcommand{\plt}[1]{\authornote{magenta}{TPLT}{#1}} %For explanation of the template
\newcommand{\an}[1]{\authornote{cyan}{Author}{#1}}

%% Common Parts

\newcommand{\progname}{ProgName} % PUT YOUR PROGRAM NAME HERE %Every program
                                % should have a name


\begin{document}

\title{User Guide for \myprogname} 
\author{Peter Michalski}
\date{\today}
	
\maketitle

\pagenumbering{roman}

\section{Revision History}

\begin{tabularx}{\textwidth}{p{4cm}p{2cm}X}
\toprule {\bf Date} & {\bf Version} & {\bf Notes}\\
\midrule
Dec. 12, 2019 & 1.0 & Initial Document\\
\bottomrule
\end{tabularx}

~\newpage

\tableofcontents

~\newpage

\listoftables

~\newpage

\section{Symbols, Abbreviations and Acronyms}

Please see Section \ref{ext1-CASYMBOLS} and Section \ref{ext1-CAABBACR} of the Commonality Analysis (\citet{LBM_CA_PM}).

~\newpage

\pagenumbering{arabic}

\section{General Information}
\label{generalinfo}

\subsection{Purpose}
\label{purpose}

\noindent The purpose of this document is to help users of {\myprogname}.

\subsection{Scope}

The document has separate sections that address usage of the system and references which arise in other documents of {\myprogname}.

~\newpage

\section{Quick Start}
\label{quickstart}
This section outlines how to use {\myprogname}.

\subsection{Environment Setup}

The following python libraries need to be installed on the system:
\begin{enumerate}
	\item datetime
	\item logging
	\item matplotlib.pyplot
	\item numpy
	\item os
	\item pylbm
	\item sumpy
	\item sys
	\item time
\end{enumerate}

\noindent Depending on your Python version, many of these should already be available. \\

\noindent pyLBM can be installed in the following ways:

\begin{enumerate}
	\item with conda:
	
	conda install pylbm -c conda-forge
	
	\item with pypi:
	
	pip install pylbm
	
	\item from source:
	
	git clone https://github.com/pylbm/pylbm
\end{enumerate}

\subsection{Using {\myprogname}}
\label{usingprog}

\subsubsection{Inputs}
\label{inputs}
The inputs.txt file shall be placed into the Input directory found within the src directory.\\


\noindent The inputs of the input.txt file shall be structured in the following manner:\\ 

Key Value

Key Value

etc..\\

For example:\\

Library 1

Problem 1\\

\noindent The keys required for a Von Karman Vortex Street problem (Problem 1) using pyLBM (Library 1) are:
\begin{enumerate}
	\item Dimensions
	\item VelocityDirections
	\item ReynoldsNumber
	\item Density
	\item BulkViscosity
	\item Time
	\item Size
\end{enumerate}


\noindent The acceptable range of inputs can be found in Table \ref{table:fstageiminrange} of Section \ref{inrange}. 

\subsubsection{Running {\myprogname}}

Run M2SystemControl.py to run {\myprogname}.

\subsection{Error Messages}

The following are error messages that reference this User Guide:

\begin{enumerate}
	\item MssngProb : The input.txt file is missing (or has incorrect) required parameters for the designated problem. Please see the User Guide.\\
	
	Required parameters (keys) for the designated problem can be found in Section \ref{inputs}.\\
	
	\item OuBounds: The input file parameter X is out of bounds. Please see the
	User Guide.\\ 
	
	Allowable input bounds can be found in Table \ref{table:fstageiminrange} in Section \ref{inputkeymap}.\\
	
	\item UnknwnParm: The parameter X is not known to the system. Please see
	the User Guide.\\
	
	Correct parameters (keys) can be found in the column ``\textbf{Correct Input File Key}'' of Table \ref{table:InputMapping} in Section \ref{inputkeymap}.\\
\end{enumerate}

~\newpage

\section{References from Other Documents}
\label{otherdocs}
This section addresses references to this document from other documents of {\myprogname}.
	
\subsection{MIS}

This section addresses references from the MIS (\citet{LBM_MIS_PM}).\\

Libraries available in the first implementation:
\begin{enumerate}
	\item pyLBM
\end{enumerate}

Problems solved in the first implementation:
\begin{enumerate}
	\item Von Karman Vortex Street
\end{enumerate}

\subsection{System VnV}

This section addresses references from the System VnV (\citet{LBM_SVNV_PM}).
Please see Section \ref{usingprog}.


\subsection{Unit VnV}

This section addresses references from the Unit VnV (\citet{LBM_UVNV_PM}).
Please see Section \ref{usingprog}.

~\newpage
\clearpage
\bibliographystyle {plainnat}
\bibliography {../../refs/References}

~\newpage
\section{Appendix}

\subsection{Input Key Mapping}
\label{inputkeymap}

\begin{table}[!h]
	\begin{center}
		\begin{tabular}{| c | c | c |}
			\hline
			\textbf{System VnV Symbol} & \textbf{Unit VnV Symbol} & \textbf{Correct Input File Key}\\
			\hline
			$\eta_b$&BulkViscosity & BulkViscosity \\
			\hline
			$\rho$&Density & Density\\
			\hline
			$\mathrm{D}$&Dimensions & Dimensions\\
			\hline
			Library&Library & Library\\
			\hline
			Problem&Problem & Problem\\
			\hline
			$Re$&ReynoldsNumber & ReynoldsNumber\\
			\hline
			$\eta_s$&ShearViscosity & ShearViscosity\\
			\hline
			Size&Size & Size\\
			\hline
			$t$&Time & Time\\
			\hline
			$\mathrm{Q}$&VelocityDirections & VelocityDirections\\
			\hline
		\end{tabular}
		\caption{First Stage Implementation Input Mapping}
		\label{table:InputMapping}
	\end{center}
\end{table} 

~\newpage
\subsection{Input Range}
\label{inrange}
\begin{table}[!h]
	\begin{center}
		\begin{tabular}{| c | c | c |}
			\hline
			\textbf{Input Key} & \textbf{Minimum Value} & \textbf{Maximum Value}\\
			\hline
			BulkViscosity & 0.0001 & 20000 \\
			\hline
			Density & 0.0708 & 13.6 \\
			\hline
			Dimensions & 2 & 2 \\
			\hline
			Library & 1 & 1\\
			\hline
			Problem & 1 & 1\\
			\hline
			ReynoldsNumber & 0.0001 & 50000\\
			\hline
			ShearViscosity & 0.001 & 20000\\
			\hline
			Size & 1 & 1\\
			\hline
			Time & 1 & N/A\\
			\hline
			VelocityDirections & 9 & 9\\
			\hline
		\end{tabular}
		\caption{First Stage Implementation Input Range}
		\label{table:fstageiminrange}
	\end{center}
\end{table}

\end{document}